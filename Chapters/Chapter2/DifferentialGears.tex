\subsubsection{Differential gears} \label{sec:Differentialgears}
The differential gear makes it possible to ensure that a minimal of the energy received from the motor is lost when the vehicle is turning. If one of the driving wheels brakes, the differential gears transfer the rotational energy from the braking side to the other side. This will minimize the lose of energy. The extra velocity, sent to the opposite belt, will make the vehicle turn.

A differential gear system can be seen on \figref{diffGearLight}.

\begin{figure}[H]
	\centering
	\includegraphics[scale=0.7]{figures/diffGearLightGray.pdf}
	\caption{Illustration of a differential gear system. On the left, the resistance on each side is equal, so the spider gear does not rotate(2). On the right, the resistance on the left side is bigger than the one on the right side, so the spider gear rotates and transfers more rotational energy to the right side. \cite{MechanicalEngineering}}
	\label{diffGearLight}
\end{figure}

The differential gear system contains a ring gear (1), a spider gear (2), two side gears connected to the rest of the system (3) and a pinion gear (not shown on picture) connected to the ring gear, that transfer the power from the motor to the system.

When the motor is running, the pinion gear transfer a rotational energy to the ring gear. The spider gear, fixed on to the ring gear, begins to rotate around the side gear. If the resistance on both side gears is the same, the same power is needed to rotate the gears. Therefore the spider gear will not rotate around its own axis and apply the same rotational energy to each side gear.

When the resistance on both sides is not the same, e.g. if one of the side brakes, the spider gear will rotate around is own axis. There is a bigger resistance on one side than on the other, so the side which is not braking will be easier to rotate. The ring gear and the spider gear are rotating at the same velocity around the two side gears, but the spider gear will apply less rotational energy on the braking side than on the other side, because of it own rotation. In the case that the resistance on one side is infinite, as the spider gear is rotating at the same speed that the ring gear, the side gear on the none braking side will rotate twice as fast, with the assumption that there is no loss in the system.

Instead of only having one spider gear, as seen in \figref{diffGearLight}, there is two on the received vehicle, this will give more reliability and solidity. 

%For the differential gear system on the vehicle, there is a more complete setup, shown on \figref{diffGearFull}
	%\caption{Illustration of the differential gear system on the vehicle \cite{MechanicalEngineering}}
%\begin{minipage}{\linewidth}
%      \centering
%      \begin{minipage}{0.65\linewidth}
%          \begin{figure}[H]
%              \includegraphics[width=0.95\textwidth]{figures/diffGearFull}
%              \caption{Illustration of the differential gear system placed on the received vehicle \cite{MechanicalEngineering}}
%              \label{diffGearFull}
%          \end{figure}
%      \end{minipage}
%      \hspace{0.05\linewidth}
%      \begin{minipage}{0.25\linewidth}
%      		\begin{enumerate}
%      			\item Pinion gear
%      			\item Ring gear
%      			\item Spider gears
%      			\item Side gears
%      		\end{enumerate}
%      \end{minipage}
%  \end{minipage}




%\begin{table} [H]
%\begin{tabular}{p{3cm} p{3cm}}

%\begin{figure}
%	\includegraphics[scale=0.7]{figures/diffGearFull}
%	\label{diffGearFull}
%\end{figure}

%&

%\begin{enumerate}
 % \item Hallo
%\end{enumerate}

%\end{tabular}
%\end{table}

%\begin{figure} [h]
%	\centering
%	\includegraphics[scale=0.7]{figures/diffGearFull}
%	\caption{Illustration of the differential gear system on the vehicle \cite{MechanicalEngineering}}
%	\label{diffGearFull}
%\end{figure}

%The functionality of the differential gear is the same as with one spider gear. A gear shaft with is connected to the gear which gets rotated by the motor, is transferring the rotation from the motor to the ring gear(1). The spider gears(2) fixed on the ring gear rotates the side gears(3), in respect to the resistance affecting the side gears.

%\subsection{Things that maybe should be placed in this section}

%When a permanent magnet DC motor is used, the inductor and resistor in the motor causes a time constant, that the PWM time period should be less than:

%\begin{flalign}
%T &< 2 \cdot \frac{L_a}{R_a} \cdot ln(1-\frac{P}{100})\unit{s}
%\end{flalign}

%Where P is the duty cycle in percent, and La is the inductance in the motor, and Ra is the resistance in the motor.
