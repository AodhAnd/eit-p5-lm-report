\subsection{Differential gears}

When the servomotor is braking, it triggers the differential gears to make the vehicle turn.\\
Differential gears are used to reduce slipping between opposite wheels when a car is turning. The system can not avoid slipping under the tracks because they are too long, they will rotate at least around the center of the track. However, using differential gears will help control the course of the vehicle by braking one of the tracks and transfering some of its power to the other track.\\

The system of differential gears is simple. As seen on \figref{diffGearLight}, the power is transmitted from the pinion gear (not represented on this picture) to the spider gear (here in green) fixed on the ring gear (here in blue). Only the spider gear is connected to the side gears(here in pink and yellow), fixed to the wheels.\\

\begin{figure}[H]
	\centering
	\includegraphics[scale=0.7]{figures/diffGearLight}
	\caption{Transfer of the power to the side gear when the spider gear is blocked vs when it spins \cite{MechanicalEngineering}}
	\label{diffGearLight}
\end{figure}

When the ring gear is turning but the spider gear does not spin, both side gears turn the same speed, but if one side gear is blocked, the spider gear will spin,  and the other side gear will turn faster.\\

There are usually two spider gears (here in red) for more reliability and solidity, and the ring gear (here in blue) is set in motion by the pinion gear (here in brown in front) as seen on \figref{diffGearFull}.\\

\begin{figure}[H]
	\centering
	\includegraphics[scale=0.7]{figures/diffGearFull}
	\caption{Description of the differential system going straight vs turning \cite{MechanicalEngineering}}
	\label{diffGearFull}
\end{figure}

This is the element the servo is using when braking one track, allowing the other track to turn still.\\\\



The description of the vehicle allows then to consider a prototype of the functionnalities needed in this project.


