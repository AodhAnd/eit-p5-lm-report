\subsection{Servo}
On the vehicle there is a servo, which is a S3003 by Futaba. \todo{Insert ref to the datasheet}
The servo is used for the steering of the vehicle. The servo controls the steering through breaking one of the belts, in part 3 on \figref{} in \secref{}, and transfer that power over to the other belt, thanks to the differential gear box.

The Servo is controlled by a PWM signal from the arduino card. The received PWM signal is converted into an angle by the servo to control the steering of the belts, as seen on \figref{timeVSangle}.

A mechanical arm is mounted on the servo and connected to the brakes of the tracks. When the servo rotates one way, it triggers the brakes more on one side and stop breaking at the other side, according to the value of the angle sent, through the PWM signal. This way, one of the belt will drive slower and the other belt drive faster, by the help of the differential gear system, which will make the vehicle turn, without decreasing the output power of the system.

\begin{figure}[H]
	\centering
	\includegraphics[scale=0.6]{figures/TimeVSangle.pdf}
	\caption{Convertion from time to angle by the servo}
	\label{timeVSangle}
\end{figure}

The servo reacts linearly to the PWM signal and the cycle of the signal is 30 milliseconds. To get the servo to be in neutral position at 90°, the servo needs a PWM signal of 4.83 \% (a period of 1450 microseconds). For 180°, it is 8.33 \% (a period of 2500 microseconds) and 1.67 \% (a period of 500 microseconds) for 0°. \\\todo{Insert ref to the datasheet}

When the servo brakes on one of the belt, the force, that should be lost in the breaking, is transfer to the other belt, by the differential gear system.