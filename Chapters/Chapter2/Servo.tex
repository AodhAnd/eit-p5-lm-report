\subsubsection{Servo} \label{Servo}
On the vehicle there is a servo, which is a S3003 by Futaba\cite{futaba}.
The servo is used for steering of the vehicle. The way the servo turn the vehicle, is by braking the gears on one side, and transferring the velocity over to the other side, by utilizing the differential gears.
%
\todo{Please add voltage requirements, needs to be referenced to later}
The Servo is controlled by a PWM signal from a controller. By applying a specific PWM signal the servo turns to a certain angle (See \tableref{timeVSangle}). When the arm rotates one way, it triggers the brake on that side, but does not affect the other side.
%
\begin{table}[H]
\centering
\begin{tabular}{|r|l|l|}
\hline%--------------------------------------------------------------------------------
  \textbf{Angle}       &  \textbf{Pulse Width   } \\
\hline%--------------------------------------------------------------------------------
  \si{0^{\circ}\ \ }   &  \si{500\ \mu s}         \\
\hline%--------------------------------------------------------------------------------
  \si{90^{\circ}\ \ }  &  \si{1450\ \mu s}        \\
\hline%--------------------------------------------------------------------------------
  \si{180^{\circ}\ \ } &  \si{2500\ \mu s}        \\
\hline%--------------------------------------------------------------------------------
\end{tabular}
\caption{Given PWM in relation to servo angle for a signal of period \si{30000 \mu s}}
\label{timeVSangle}
\end{table}
%
The servo reacts linearly to the PWM signal with period of 30000 \si{\mu s}\cite{futaba}. The servo can only affect one brake at a time. When the servo is affected by a pulse of 500 \si{\mu s} it will turn to its maximum position at one side, thus braking the right belt. When given a pulse width of 2500 \si{\mu s}, the servo turns to its maximum position at the other side, thus braking the left belt. When applying a pulse width of 1450 \si{\mu s} the servo turns to neutral position, \si{90^{\circ}}. In neutral position the servo does not pull on either of the brakes.