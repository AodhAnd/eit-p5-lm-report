\subsection{Motor Modelling}\label{motormodelling}
In this section a model of a motor is depicted only as an electrical system, which delivers a rotational force called torque, \si{\tau_m}. The electrical model provides the motor's produced torque to the drivetrain which is further discussed in \secref{DriveTrain}.
%
\subsubsection{Electrical Model}
The output needed from the motor's electrical model is the torque, \si{\tau_m}. To obtain the torque, the formula for translating the electrical current, \si{i_a}, to torque is utilized:
%
\begin{flalign}\centering
  \eq{\tau_m(t)}{K_t \cdot i_a(t)} \unit{N \cdot m}
  \label{equ:motortorque}
\end{flalign}
\hspace{6mm} Where:\\
\begin{tabular}{p{1cm}lll}
& \si{\tau_m(t)} & is the rotational force torque &\unitWh{N \cdot m} \\
& \si{i_a(t)} & is the electrical closed loop current &\unitWh{A}\\
& \si{K_t} & is the motor constant &\unitWh{N \cdot m \cdot A^{-1}}
\end{tabular}

An expression for the current, \si{i_a(t)}, is required to derive a model for the electrical system. In \figref{fig:electricaldiagrammotor} an electrical diagram of the motor is displayed.
%
\begin{figure}[H]
\centering
	\begin{circuitikz}[american voltages]
		\draw
		
		% electromotive force 
		(0,0) to [short] (6,0)
		%to [sV, l=$e_b$] (6,2) %  voltage
		(6,0) to [V, l=$e_b$] (6,3)
		%to node[short]{}(6,2)

		%to node[short]{}(0,0)		 
		(0,0) to [V, l=$U_a$] (0,3) %  voltage

		
		%to [R, l=$Z_G$] (3,3) % generator impedance
		
		(0,3) to [R, l_=$R_a$, i>_=$i_a$] (3,3)	
		
		to [L, l_=$L_a$] (6,3); 
	\end{circuitikz}
  \caption{A electrical diagram of the motor}
  \label{fig:electricaldiagrammotor}
\end{figure}
%
By using Kirchoff voltage law on the closed loop, seen in \figref{fig:electricaldiagrammotor}, an expression including $i_a$ can be derived:
%
\begin{flalign}
\eq{U_a(t)}{R_a \cdot i_a(t) + L_a \cdot \frac{di_a(t)}{dt} + e_b(t)}\unit{V} 
\label{MotorClosedLoop}
\end{flalign}
\hspace{6mm} Where:\\
\begin{tabular}{p{1cm}lll}
& \si{U_a(t)} & is the supply voltage                       &\unitWh{V} \\
& \si{R_a}    & is the internal resistance in the motor     &\unitWh{\Omega}\\
& \si{L_a}    & is the inductance in the motor              &\unitWh{H} \\
& \si{e_b}    & is the electromotive force, also called EMF &\unitWh{V} \\
\end{tabular}

The electromotive force, \si{e_b(t)}, is equivalent to:
%
\begin{flalign}
\eq{e_b(t)}{K_e \cdot \dot{\theta}_m(t)}\unit{V} 
\end{flalign}
\hspace{6mm} Where:\\
\begin{tabular}{p{1cm}lll}
& \si{K_e}            & is the electromotive constant        &\unitWh{Wb} \\
& \si{\dot{\theta}_m(t)} & is the angular velocity in the motor &\unitWh{rad \cdot s^{-1}} \\
\end{tabular}

The equivalent for the electromotive force is substituted into \eqref{MotorClosedLoop}.
%
\begin{flalign}
\eq{U_a(t)}{R_a \cdot i_a(t) + L_a \cdot \frac{di_a}{dt} + K_e \cdot \dot{\theta}_m(t)}&
\end{flalign}
%
The Laplace transform is applied to the derived equation:
%
\begin{flalign}
\eq{U_a(s)}{R_a \cdot i_a(s) + L_a \cdot s \cdot i_a(s) + K_e \cdot \omega_m(s)}&
\end{flalign}
%
The equation is solved for \si{i_a}:
%
\begin{flalign}
\eq{i_a(s)}{\frac{U_a(s) - K_e \cdot \omega_m(s)}{L_a \cdot s + R_a}}&
\end{flalign}
%
By substituting the derived equation for $i_a$ into \eqref{equ:motortorque}, a new expression for the motor's torque is derived. 
%
\begin{flalign}
\eq{\tau_m}{K_t \cdot i_a(s)}&\nonumber\\
\eq{\tau_m}{K_t \cdot \frac{U_a(s) - K_e \cdot \omega_m(s)}{L_a \cdot s + R_a}}
  \label{eq:Totaltorquewithcurrentexpression}
\end{flalign}
%
By dividing with the voltage applied to the motor, $U_a$, a relation can be established:
%
\begin{flalign}
\eq{\frac{\tau_m}{U_a(s) - K_e \cdot \omega_m(s)}}{\frac{K_t}{L_a \cdot s + R_a}}&
  \label{eq:TotaltorquewithcurrentexpressionTransferFunction}
\end{flalign}
%
A relation for the electrical model relative to the motor's torque has been derived. Thus enabling setting up calculations for the vehicle's drivetrain.