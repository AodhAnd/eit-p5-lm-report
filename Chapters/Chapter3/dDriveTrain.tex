\subsection{Drivetrain}\label{DriveTrain}
The drivetrain translates the torque $\tau_m$, given by the motor, into the actual movement of the vehicle. A mechanical depiction of the parts considered in this modeling section can be seen on \figref{fig:DrivetrainMechanicalModel}, where the input torque appears on the motor shaft and translates through the system ending up as a linear velocity of the vehicles mass on the belts, i.e. the velocity of the vehicle. The velocity model of the drivetrain is created by only considering when the vehicle's trajectory follows a straight line.\\\\

\begin{figure}[H]
	\centering
	\includegraphics[scale=0.8]{figures/mechanicalDrawingSystem.pdf}
	\caption{A diagram showing the outlines of the mechanical model of the drivetrain considered in this section}
	\label{fig:DrivetrainMechanicalModel}
\end{figure}

The modeling of the drivetrain is limited to only modeling parts of the total drivetrain system, to make it more manageable.

\subsubsection{Model of the Internal Gears}\label{BlackBoxModel}
To get a rough approximation of the effect of the drivetrain, a black boxed model is used. The black box in the model is placed between box 1 and including box 3 seen in \chapref{sec:Vehicledescription} \figref{vehicleDescriptionDriveTrain}. The gear, $G_m$, connected to the motor shaft, is connected to the gear, $G_d$, which represents the gears and shafts that has been black boxed, where the output appears on the shaft connected to the drive-wheel, $G_t$, see \figref{fig:DrivetrainMechanicalModel}. The number of teeth on the gear, $G_d$, represents the ratios throughout the drivetrain, such that the gear ratio between $G_m$ and the output at the drive gear is given by: $\frac{N_m}{N_d}$, where $N_m$ is the number of teeth on the motor gear and $N_d$ is the calculated number of teeth on the black box gear.

The torque of the motor along with the contributions from the load, i.e. the drivetrain, affecting the motor shaft, is depicted in as a free body diagram in \figref{fig:MotorGearFreeBodyDiagram}.

\begin{figure}[H]
	\centering
	\includegraphics[scale=1.2]{figures/freeBodyMotorGear.pdf}
	\caption{A free body diagram of the motor gear, $G_m$}
	\label{fig:MotorGearFreeBodyDiagram}
\end{figure}

The radius of the motor gear times the contact force, $r_d \cdot f_c$, is the torque translated from the motor gear, $G_m$, to the gear representing the drivetrain, $G_d$, in \figref{fig:BlackBoxGearFreeBodyDiagram}. Thus it appears as an opposing torque to the applied motor torque, $\tau_t$. From \figref{fig:MotorGearFreeBodyDiagram}, the following equation can be derived:
 \begin{flalign}
   \eq{J_m \cdot \dot{\omega}_m(t)} {\tau_m(t) - B_m \cdot \omega_m(t) - r_m \cdot f_c(t)}\unit{N \cdot m}\nonumber
   \label{eq:MotorGearNewtonSecLaw}
 \end{flalign}
%
\hspace{6mm} Where:\\
\begin{tabular}{ p{1cm} l l l}
& $J_m$ 						& is the motor's inertia                                         &\unitWh{kg \cdot m^2} \\
& $\omega_m$        & is the angular velocity of the motor                           &\unitWh{rad \cdot s^{-1}} \\
& $\dot{\omega}_m$ 	& is the angular acceleration of the motor                       &\unitWh{rad \cdot s^{-2}} \\
& $\tau_m$ 			    & is the torque delivered by the motor                           &\unitWh{N \cdot m} \\
& $B_m$             & is the motor's friction coefficient                            &\unitWh{N \cdot m \cdot s \cdot rad^{-1}} \\
& $r_m$             & is the radius of the gear, $G_m$, connected to the motor shaft &\unitWh{m} \\
& $f_c$							& is the contact force between the two gears                     &\unitWh{N}
\end{tabular}

To describe the translations of the torques throughout the system an other free body diagram is applied in \figref{fig:BlackBoxGearFreeBodyDiagram}, showing the system's effects on the blackbox gear, $G_d$, giving the output on the drive gear.

\begin{figure}[H]
	\centering
	\includegraphics[scale=1]{figures/freeBodyDriveGear.pdf}
	\caption{A free body diagram of the `black box' gear, $G_d$}
	\label{fig:BlackBoxGearFreeBodyDiagram}
\end{figure}

An equation is extracted from \figref{fig:BlackBoxGearFreeBodyDiagram}:
\begin{flalign}
\eq{J_d \cdot \dot{\omega}_d(t)}{r_d \cdot f_c(t) - r_t \cdot f_b(t)} \unit{N\cdot m}
\label{eq:BlackBoxGearNewtonSecLaw1}
\end{flalign}
\hspace{6mm} Where:\\
\begin{tabular}{p{1cm}lll}
& $J_d$ 						& is the `black box' gear inertia                                    &\unitWh{kg \cdot m^2} \\
& $\dot{\omega}_d$ 	& is the angular acceleration of the `black box' gear, $G_d$         &\unitWh{rad \cdot s^{-2}} \\
& $r_d$             & is the radius of the `black box' gear, $G_d$                       &\unitWh{m} \\
& $r_t$             & is the radius of the drive wheel, $G_t$ (translational)            &\unitWh{m} \\
& $f_b$             & is the coefficient of the contact force between $G_d$ and the belt &\unitWh{N} \\
& $f_c$						  & is the contact force between the two gears                         &\unitWh{N}
\end{tabular}

Two equations for the internal gears have been found, \eqref{eq:MotorGearNewtonSecLaw} and \eqref{eq:BlackBoxGearNewtonSecLaw}. In the following paragraph a model of the translation between the drive-gear and the belt will be created.

%% SSSECTION : BELT MODEL %%
\subsubsection{Model of the Vehicle's acceleration}\label{BeltModel}
It is necessary to find the translation between the drive-gear and the belt to be able to describe the movement of the vehicle. Furthermore, it is needed to be able to make a full model of the system, going from the gear connected to the motor shaft to the vehicle's belts. In \figref{fig:BeltMechanicalDiagram} a mechanical diagram illustrates how the force given through the drive wheel is translated by the belts to a linear force acting on the mass which in turn yields the vehicle's velocity.

\begin{figure}[H]
	\centering
	\includegraphics[scale=0.8]{figures/mechanicalDrawingBelt.pdf}
	\caption{A mechanical diagram of the drive-gear rotating the belt affected by the vehicle's mass}
	\label{fig:BeltMechanicalDiagram}
\end{figure}

By using \figref{fig:BeltMechanicalDiagram} a free body diagram of the mass, $M$, and the forces applied to it is derived.

\begin{figure}[H]
	\centering
	\includegraphics[scale=.8]{figures/freeBodyBelt.pdf}
	\caption{A free body diagram illustrating the vehicle's mass affected by the surrounding forces}
	\label{fig:BeltFreeBodyDiagram}
\end{figure}

From \figref{fig:BeltFreeBodyDiagram}, a mechanical equation of the vehicle's resulting force is found to be:\todo{In free body diagram: remove rt, the forces should be in N, not N*m}
%
\begin{flalign}
\eq{M \cdot \dot{v}(t)}{f_b(t) - B_{sys} \cdot v(t)} \unit{N}
\label{eq:BeltMassNewtonSecLaw}
\end{flalign}
\hspace{6mm} Where:\\
\begin{tabular}{p{1cm}lll}
& $M$ 			  & is the vehicle's total weight                                        &\unitWh{kg} \\
& $v$        	& is the linear velocity of the vehicle                                &\unitWh{m \cdot s^{-1}} \\
& $\dot{v}$ 	& is the linear acceleration of the vehicle                            &\unitWh{m \cdot s^{-2}} \\
& $B_{sys}$   & is the coefficient of the friction throughout the rotational parts   &\unitWh{N \cdot s \cdot m^{-1}} \\
\end{tabular}

In the following section the three mechanical equations, \eqref{eq:MotorGearNewtonSecLaw}, \eqref{eq:BlackBoxGearNewtonSecLaw1} and \eqref{eq:BeltMassNewtonSecLaw} will be combined.

\subsubsection{Model of the Combined Drivetrain}\label{DrivetrainModeling}
In this section a model will be created of the combined drivetrain, by assembling the derived mechanical equations, \eqref{eq:MotorGearNewtonSecLaw}, \eqref{eq:BlackBoxGearNewtonSecLaw1} and \eqref{eq:BeltMassNewtonSecLaw}. This allows to get a linear velocity of the vehicle, $v$, from the torque, $\tau_m$, delivered by the motor. The three equations are linked through the contact forces $f_c$ and $f_b$.\\\\
%
The Laplace-transform is applied on the three equations in order to be able to deduce a transfer function of the input torque, $\tau_m(s)$, to the output velocity, $V(s)$:
%
\begin{flalign}
\eq{J_m \cdot \omega_m(s) \cdot s}{\tau_m(s) - B_m \cdot \omega_m(s) - r_m \cdot F_c(s)}&
\label{eq:MotorGearNewtonSecLawLaplace}
\end{flalign}
%
\begin{flalign}
\eq{J_d \cdot \omega_d(s) \cdot s}{r_d \cdot F_c(s) - r_t \cdot F_b(s)}&
\label{eq:BlackBoxGearNewtonSecLawLaplace}
\end{flalign}
%
\begin{flalign}
\eq{M \cdot V(s) \cdot s}{F_b(s) - B_{sys} \cdot V(s)}&
\label{eq:BeltMassNewtonSecLawLaplace}
\end{flalign}

$F_c$ is isolated from \eqref{eq:MotorGearNewtonSecLawLaplace}:
\begin{flalign}
\eq{F_c(s)}{\frac{\tau_m(s) - B_m \cdot \omega_m(s) - J_m \cdot \omega_m(s) \cdot s}{r_m}}&
\label{eq:GearsContactForceMotorGearLaplace}
\end{flalign}
%
$F_b$ is isolated from \eqref{eq:BlackBoxGearNewtonSecLawLaplace}:
\begin{flalign}
\eq{F_b(s)}{\frac{r_d \cdot F_c(s) - J_d \cdot \omega_d(s) \cdot s}{r_t}}&
\label{eq:GearsContactForceBlackBoxGearLaplace}
\end{flalign}
%
$F_b(s)$ is isolated from \eqref{eq:BeltMassNewtonSecLawLaplace}.
\begin{flalign}
\eq{F_b(s)}{V(s) \cdot (M \cdot s - B_{sys})}&
\label{eq:BeltContactForceLaplace}
\end{flalign}

The $F_c(s)$ in \eqref{eq:GearsContactForceBlackBoxGearLaplace} is substituted for the expression given by \eqref{eq:GearsContactForceMotorGearLaplace}:
\begin{flalign}
\eq{F_b(s)}{\frac{r_d \cdot (\frac{\tau_m(s) - B_m \cdot \omega_m(s) - J_m \cdot \omega_m(s) \cdot s}{r_m}) - J_d \cdot \omega_d(s) \cdot s}{r_t}}&\nonumber\\
\eq{F_b(s)}{\frac{\tau_m(s) \cdot r_d}{r_m\cdot r_t} - \frac{\omega_m(s)\cdot r_d}{r_m\cdot r_t}\cdot (B_m + J_m \cdot s) - \frac{\omega_d(s)}{r_t}\cdot J_d\cdot s}&
\label{eq:BlackBoxGearNewtonSecLawLaplace&MotorGearNewtonSecLawLaplace}
\end{flalign}

The two expressions for $F_b(s)$ in \eqref{eq:BlackBoxGearNewtonSecLawLaplace&MotorGearNewtonSecLawLaplace} and \eqref{eq:BeltContactForceLaplace} respectively are coupled:
\begin{flalign}
\eq{V(s) \cdot (M \cdot s - B_{sys})}{\frac{\tau_m(s) \cdot r_d}{r_m\cdot r_t} - \frac{\omega_m(s)\cdot r_d}{r_m\cdot r_t}\cdot (B_m + J_m \cdot s) - \frac{\omega_d(s)}{r_t}\cdot J_d\cdot s}&
\label{eq:BlackBoxGearNewtonSecLawLaplace&MotorGearNewtonSecLawLaplace&&BeltContactForceLaplace}
\end{flalign}

Using the relationship of ratios to relate the two rotational velocities as a step towards converting all rotational velocities to linear velocities:
\begin{flalign}
\eq{\frac{r_m}{r_d}}{\frac{\omega_d(t)}{\omega_m(t)}}\nonumber \unit{\cdot}\\
\eq{\omega_d(t)}{\frac{r_m}{r_d}\cdot \omega_m(t)} \unit{rad\cdot s^{-1}}
\label{eq:ratioes}
\end{flalign}

Converting $\omega_d(t)$ to linear velocity and transforming to Laplace domain:
\begin{flalign}
\eq{v(t)}{\omega_d(t) \cdot 2 \cdot \pi \cdot r_t}\nonumber \unit{m\cdot s^{-1}} \phantom{(3.23   3)}\\
\eq{\omega_d(t)}{\frac{v(t)}{2\cdot \pi \cdot r_t}} \ \xRightarrow{\mathcal{L}} \ \omega_d(s) = \frac{V(s)}{2\cdot \pi \cdot r_t}&
\label{eq:omegaDtoLinearLaplace}
\end{flalign}

Converting $\omega_m(t)$ to linear velocity using \eqref{eq:ratioes} and \eqref{eq:omegaDtoLinearLaplace} and transforming to Laplace domain
\begin{flalign}
\eq{\frac{r_m}{r_d}\cdot \omega_m(t)}{\frac{v(t)}{2\cdot \pi \cdot r_t}}\nonumber \unit{rad\cdot s^{-1}} \phantom{(3.23   3)}\\
\eq{\omega_m(t)}{\frac{v(t)\cdot r_d}{2\cdot \pi \cdot r_t\cdot r_m}}  \ \xRightarrow{\mathcal{L}} \ \omega_m(s) = \frac{V(s)\cdot r_d}{2\cdot \pi \cdot r_t\cdot r_m}
\label{eq:omegaMtoLinearLaplace}
\end{flalign}

Using the previously derived from \eqref{eq:BlackBoxGearNewtonSecLawLaplace&MotorGearNewtonSecLawLaplace&&BeltContactForceLaplace}, \eqref{eq:omegaDtoLinearLaplace} and \eqref{eq:omegaMtoLinearLaplace} the final transferfunction descriping the drivetrain, including the mechanical aspect from the motor, can be constructed:

\begin{flalign}
\eq{V(s) \cdot (M \cdot s - B_{sys})}{\frac{\tau_m(s) \cdot r_d}{r_m\cdot r_t} - \frac{\frac{V(s)\cdot r_d}{2\cdot \pi \cdot r_t\cdot r_m}\cdot r_d}{r_m\cdot r_t}\cdot (B_m + J_m \cdot s) - \frac{\frac{V(s)}{2\cdot \pi \cdot r_t}}{r_t}\cdot J_d\cdot s}&\\
\eq{\frac{V(s)}{\tau_m(s)}}{\frac{\frac{1}{ \frac{r_m\cdot r_t}{r_d} \cdot B_{sys} + \frac{r_d}{2\cdot \pi \cdot r_m \cdot r_t} \cdot B_m }}{\frac{ \frac{r_m\cdot r_t}{r_d} \cdot M + \frac{r_d}{2\cdot \pi \cdot r_m \cdot r_t} \cdot J_m + \frac{r_m}{2\cdot \pi \cdot r_t \cdot r_d} \cdot J_d }{\frac{r_m\cdot r_t}{r_d} \cdot B_{sys} + \frac{r_d}{2\cdot \pi r_tm \cdot r_t} \cdot B_m} \cdot s + 1}}&
\end{flalign}

The mechanical part of the system is described by this transfer function, which in the following section is used in collaboration with the electrical model of the motor to describe the full velocity model.