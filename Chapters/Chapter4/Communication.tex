\chapter{Communication}
A communication system is desired for transporting the GoT data, calculated to coordinates on the computer, to the microcontroller, located on the vehicle. Furthermore a protocol handling packet transitions is necessary to implement on top of the communication system to be able to fulfil requirements set in \secref{sec:Requirements}.

% The har two Xbee radio modules. The communication setup for the Xbee modules will be explained in this chapter.

\subsection{OSI model}
The Open Systems Interconnection, OSI, model is a standard used to describe how information moves in a network. Each layer describing standardization of what should occur in the specific layer. The seven layers of the OSI model can be seen in \tableref{tab:OSIModel}.

\begin{table}[H]\centering
\begin{tabular}{|p{1.5cm}|p{3cm}|}
\hline%-----------------------------------------------------------------------------------------------------------------
  \textbf{Layer} & \textbf{OSI} \\
\hline%-----------------------------------------------------------------------------------------------------------------
    7 &    Application      \\
\hline%-----------------------------------------------------------------------------------------------------------------
    6 &    Presentation      \\
\hline%-----------------------------------------------------------------------------------------------------------------
    5 &    Session       \\
\hline%-----------------------------------------------------------------------------------------------------------------
    4 &    Transport    \\
\hline%-----------------------------------------------------------------------------------------------------------------
    3 &   Network     \\
\hline%-----------------------------------------------------------------------------------------------------------------
    2 &   Data link     \\
\hline%-----------------------------------------------------------------------------------------------------------------
    1 &    Physical     \\
\hline%-----------------------------------------------------------------------------------------------------------------
\end{tabular}
\caption{OSI model}
\label{tab:OSIModel}
\end{table}

An overall description of the main functionalities in the different layers is examined and a description of what layers are utilized for in the prototypes communication protocol, is given:

The Physical layer is the electrical/mechanical hardware layer. The layer handles bits and determines what is a binary 0 and 1. The prototype utilizes two Xbee's, one in the computer and one on the vehicle, for the lowest layer of communication.

The Data link layer functionality ensures physical addressing, error control when transferring data between node over the physical layer and access control \cite{JensMyerSlides}. The Data link layer is something the Xbee module handles to ensure a byte is send correctly \todo{Check up on that}.

The Network layers main functionality is routing across a network. This is not utilized in the protocol. Since to communication is only between two modules. 

The Transport layers handles packets and can assemble or disassemble packets if necessary. An important part of the functionality of this layer is to ensure reliable transmission of data packets.

  

 




%Notes:
%We use UDP, no connections

%OSI
%The physical layer is the Xbee
%The data link layer is halfway done in the code, where the package is split up in bytes. The adressing is done on the XBee and in the serial port libeary 
%The network layer may not be used here as we just transmit 360 degree
%The transport layer is the setup of how the package is setup
%The rest of the layers is not used.


\subsection{Transport Layer}


%is made to understand what contains each byte sent from the transmitter to the receiver. The use of a protocol for the transport layer will make sure that each end of the communication system understand how each package of bytes is build up.

The main object for the communication system is to send the coordinates from the computer to the microcontroller. A transport layer protocol is created to ensure packets can be transmitted, receive, i.e. extract the needed information from the packet, and to identify erroneous packets.

A connectionless system is chosen to be implemented, since it is one-way communication, from the computer to the microcontroller. Furthermore is retransmission not desired, since the vehicle is moving and a retransmission of a coordinate will cause a delay. Additionally, the sampling rate of the GoT system i 10 \si{Hz}, so sending the recently received coordinates would be insufficient. Therefore is erroneous packet is thrown away and acknowledgements is not needed.

måske noget med at den ligner udp?

\subsubsection{Protocol Package Structure}
A package structure is desired for containing necessary information for addressing, decrypting and error handling. The protocol has an header which contains a start byte, a destination and the length of the package. The body of the package is the data and the tail is the checksum and the end byte. The order of the package structure is illustrated in \tableref{CoorSetup}.

\begin{table}[H]\centering
\begin{tabular}{|>{\centering\arraybackslash}m{2cm}|>{\centering\arraybackslash}m{2cm}|>{\centering\arraybackslash}m{2cm}|>{\centering\arraybackslash}m{2cm}|>{\centering\arraybackslash}m{2cm}|>{\centering\arraybackslash}m{2cm}|} \\
\hline
Start Byte & Destination ID & length & Data & Checksum & End byte \\
\hline
\end{tabular}
\caption{The structure of a package}
\label{CoorSetup}
\end{table}

The destination ID is to ensure the packets which is transmitted is handled by the desired receiver. This way the transmitter can transmit to numerous receivers if multiple vehicles is utilized. The destination ID ensures the receivers can filter out packages not intended for them. The microcontroller has the destination ID 0000 0001. The length of the destination byte is set to one byte, so the destination can be read immediately when received.

It is not necessary to have a source in the header, since only one the computer is transmitting, and therefore it is not necessary for the vehicle to know where the packet is transmitted from. Even if there were more computers, as long as the packet is marked with the destination ID, the receivers is able to read it. If retransmission was utilized, in case of erroneous packages, a source is necessary to have the ability to ask for a retransmission from the sender. Since retransmission is not utilized a source is not included in the packet structure.

The length is utilized by the receiver. This will make it possible for the receiver to know the bit length of the package, and thereby know when the package should end. Each package created has the same length. The length is set to contain 7 bit, this can count up to $2^{7}-1 = 127$, which is sufficient.

\subsubsection{Data}
\todo{Insert value number into the GoT description}
The data from the GoT system are considered in the three coordinates system (X,Y,Z), measured from the position of the vehicle. Each of these coordinates can have a value from -9000 to 9000 (See \appref{GoTDescription}). To contain a number of this size, a 15 bit signed integer is needed. In these, 14 bit are used for the magnitude of the number which can go up to 16385 ($2^{14}-1$). The last bit is used to indicate if the number is positive or negative, called sign bit. The sign bit will be placed at the end of the 15 bit integer. As both the Arduino and the computer use big endians, this will also be apply to the protocol of the GoT system,. This means that the bit with the highest value will be next to the sign bit and the bit with the lowest value at the start of the bit array. This will give the bit array seen on \tableref{CoorSetup}.

\begin{table}[H]
\centering
\begin{tabular}{|>{\centering\arraybackslash}m{0.5cm}|>{\centering\arraybackslash}m{0.5cm}|>{\centering\arraybackslash}m{0.5cm}|>{\centering\arraybackslash}m{0.5cm}|>{\centering\arraybackslash}m{0.5cm}|>{\centering\arraybackslash}m{0.5cm}|>{\centering\arraybackslash}m{0.5cm}|>{\centering\arraybackslash}m{0.5cm}|>{\centering\arraybackslash}m{0.5cm}|>{\centering\arraybackslash}m{0.5cm}|>{\centering\arraybackslash}m{0.5cm}|>{\centering\arraybackslash}m{0.5cm}|>{\centering\arraybackslash}m{0.5cm}|>{\centering\arraybackslash}m{0.5cm}|>{\centering\arraybackslash}m{0.65cm}|}
\multicolumn{15}{c}{15 bits} \\
\hline
$2^0$ & $2^1$ & $2^2$ & $2^3$ & $2^4$ & $2^5$ & $2^6$ & $2^7$ & $2^8$ & $2^9$ & $2^{10}$ & $2^{11}$ & $2^{12}$ & $2^{13}$ & $+/-$ \\
\hline
\end{tabular}
\caption{Setup for the bit array, that contains one of the coordinate values.}
\label{CoorSetup}
\end{table}

The representation of the integer is called signed magnitude. Another representation that could be used is ones' complement. Here the number is inverted if the signed bit is true. But for an easier implementation into the GoT system, the signed magnitude is used. This is easier because the code will first check if the number is positive or negative. If negative, the sign bit is set and the number is multiplied by minus one, to get the magnitude. Then the magnitude will be converted into the bit array. When converting it back, it is only necessary to multiply the magnitude by minus one, if the signed bit is true.

\subsubsection{Checksum}
To make sure that the package data is received correctly, some error handling are added to the protocol. One of these is a checksum. The way the checksum is calculated, is by splitting the header and data up in parts of equal size and then adding these parts together. Then the summed bit array is inverted and this bit array is the checksum. 

As there are 60 bits in the header and the data combined (15 per coordinate, 8 for the destination and 7 for the length), these will be split up into three part of 20 bit and will give a checksum on 20 bit. Combined, the header, data and checksum will be 80 bit, which is equal to 10 byte, which means there will not be needed any filler bit.

\begin{table}[H]
\centering
\begin{tabular}{c c c c c c c c c c c}
   & Bit array  &     & Decimal &     & Bit 0-3 & Bit 4-7 & Bit 8-11 & Bit 12-15 & Bit 16-19 & Bit 20 \\
\hline
a) & Part 1     & $=$ & 123008  & $=$ & 0000 & 0001 & 0000 & 0111 & 1000 & \\
b) & Part 2     & $=$ & 351365  & $=$ & 1010 & 0001 & 0011 & 1010 & 1010 & \\
c) & Part 3     & $=$ & 729671  & $=$ & 1110 & 0010 & 0100 & 0100 & 1101 & \\
d) & Part 1+2+3 & $=$ & 1204044 & $=$ & 0011 & 0010 & 1111 & 1010 & 0100 & 1 \\
e) & Add carry  & $=$ & 155469  & $=$ & 1011 & 0010 & 1111 & 1010 & 0100 & \\
f) & Checksum   & $=$ & 893106  & $=$ & 0100 & 1101 & 0000 & 0101 & 1011 & \\
g) & Check      & $=$ & 1048575 & $=$ & 1111 & 1111 & 1111 & 1111 & 1111 & \\
\end{tabular}
\caption{Example on how to calculated the checksum. The parts contains the header, where the length is set to 96 and the coordinates 4259, 7511 and -6418. The binary number is writing up in big endians.}
\label{ChecksumExp}
\end{table}

In \tableref{ChecksumExp} there is an example of the calculation of the checksum. The three parts containing the header and data are added together (seen on line d). This gives a carry because there are only 20 bits allowed, that is added in the start of the number (seen on line e). The new number is then inverted giving the checksum (seen on line f).

The way to control the errors with the checksum, is to add the checksum together with the three 20 bit parts and add the carry on, if there are any. If the outcome of this addition is a 20 bit part, where all bits are true, the package is correctly received. If not, the control will not consider this package. The reason for the error can be a bit shift in one of the parts, either because there are bytes that have switch place or because some data have not been received. 

A flaw with this control with the checksum, is that an error cancelled out with another error can not be detected. This can happen when the system add the three parts and the checksum together. An example, is if the first bit in part one is true and the first bit in part two is false. If both of these bits are shifted independently, so the bit from part one become false and the bit in part two true, the control with the checksum will not detect this error. The chances for this case are very small as the changes have to go exactly out with each other. And by only having 4 parts of 20 bits, the chances for this to happen is smaller than if the checksum was set to 10 bit, for example, and therefore the addition would have to happen with 7 parts of 10 bit. This would give more bits that have the same rank, so there would be a higher chance for a bit changing to go out with another. This is the reason why using a bigger checksum, even if it takes more place.

\subsubsection{Start and end byte}

As the GOT system send a package each time it makes a measurement of the coordinates, a queue of package can appear in the Arduino, if it does not read fast enough. This can confuse the system which may not differenciate the package from each other. It can be avoided by adding a start byte at the start of the package and an end byte at the end. 

When the system wants to find the start of the package, it will search for the start byte. When finding this byte, it will read the header, data and checksum and then look for the end byte. If the end byte is not there, it means that the package was not received correctly and is thrown away. If the end byte is there, the system will take the whole package and apply the error handling on it. 

The start byte will be set to 0000 1111. This is chosen so that the start byte can not have the same value as one of the other bytes in the header, so these will not be misunderstood as the start byte. To have the start and end byte as different as possible, as the end byte will come just before the next start byte, the end byte is the inverted number of the start byte, which is 1111 0000.


With the data, header, checksum, start and end byte, a package will look like the illustration on \tableref{PackageLook}

\begin{table}[H]
\centering
\begin{tabular}{|c|c|>{\centering\arraybackslash}m{0.3cm}|>{\centering\arraybackslash}m{0.3cm}|>{\centering\arraybackslash}m{0.3cm}|>{\centering\arraybackslash}m{0.3cm}|>{\centering\arraybackslash}m{0.3cm}|>{\centering\arraybackslash}m{0.3cm}|>{\centering\arraybackslash}m{0.3cm}|>{\centering\arraybackslash}m{0.3cm}|>{\centering\arraybackslash}m{0.3cm}|>{\centering\arraybackslash}m{0.3cm}|>{\centering\arraybackslash}m{0.3cm}|>{\centering\arraybackslash}m{0.3cm}|>{\centering\arraybackslash}m{0.3cm}|>{\centering\arraybackslash}m{0.3cm}|>{\centering\arraybackslash}m{0.3cm}|>{\centering\arraybackslash}m{0.3cm}|}
\hline
\multicolumn{2}{|c|}{Offsets} & \multicolumn{8}{c}{Byte 1} & \multicolumn{8}{|c|}{Byte 2} \\
\hline
\multicolumn{1}{|c}{Byte} & \multicolumn{1}{|c|}{Bit} & 0 & 1 & 2 & 3 & 4 & 5 & 6 & 7 & 8 & 9 & 10 & 11 & 12 & 13 & 14 & 15 \\
\hline
0 & 0 & \multicolumn{8}{c}{Start byte} & \multicolumn{8}{|c|}{Destination} \\
\hline
2 & 16 & \multicolumn{7}{c}{Length} & \multicolumn{9}{|c|}{X coordinate} \\
\hline
4 & 32 & \multicolumn{6}{c}{X coordinate} & \multicolumn{10}{|c|}{Y coordinate} \\
\hline
6 & 48 & \multicolumn{5}{c}{Y coordinate} & \multicolumn{11}{|c|}{Z coordinate} \\
\hline
8 & 64 & \multicolumn{4}{c}{Z coordinate} & \multicolumn{12}{|c|}{Checksum} \\
\hline
10 & 80 & \multicolumn{8}{c}{Checksum} & \multicolumn{8}{|c|}{End byte} \\
\hline
\end{tabular}
\caption{Illustration of a package, that will be send from the transmitter to the receiver.}
\label{PackageLook}
\end{table}

This gives a package length of 96 bit (12 byte). As the GoT system have a sampling frequency of 10 Hertz and each sampling gives out 12 byte, the transfer speed for the Xbee have to be greater than 120 byte per second. 

\subsubsection{Error handling}
As there is a chance that the package is getting damaged under the way from the transmitter to the receiver, some error handling is used to check the packages for error and throw the damage packages away. One of these is the checksum, which is explained further up in this section. Besides the checksum, the start and end byte is used to see if the number of byte received is right, according to the length in the header.

But as with the checksum, where there is a flaw if there are multiples errors and these cancel each other out in the control, there is a flaw in the error handling for receiving packages. The receiving process can be seen on \figref{FlowReceiver}.

%\begin{figure}[H]
%\centering
%\includegraphics[scale=0.7]{figures/FlowReceiver.pdf}
%\caption{Flow chart over the error handling in the receiver part.}
%\label{FlowReceiver}
%\end{figure}

If the package is transmitted correctly, the 1st byte will be the start byte, the 2nd byte will be the destination ID of the receiver. The 3rd byte will contain 7 bit for the length and 1 bit for the first coordinate. If the length is correct, it will be 96, which is 12 byte. The last byte, the 12th one, is the end byte. Then if these condition are all correct, the receiver will make a control with the checksum. If that one is also correct, the system will convert the data and return the coordinates to the rest of the system.

If there is an error, then all the bytes that have been read will be thrown away. This is because the bytes are taken from the buffer of the receiving Xbee and can not be put back at the same place, after being taken out. So if the start byte and the destination are correct, but the end byte is not, the whole package is thrown away. If the start byte is not correct, then only that byte will be thrown away until finding a correct one. This method is applied because the system can only work with complete packages.

This prevents the receiver to start in the middle of a package, as the system will just throw away the bad package, until it finds the start byte. A problem that comes with this feature, is that as the start byte is only on a normal byte, a byte in the data or the checksum part can have the same value. If the receiver then by fault beginning looks after the start byte in these part and find a byte equal to the start byte, it can consider the 9 next bytes as a package. The problem will first happen if the byte afterwards is equal to the receiver's destination ID, and the first seven bits of the next byte are equal to 96. An example on a situation where this problem happens can be seen on \tableref{errorPro}.

\begin{table}[H]
\centering
\begin{tabular}{c | m{0.1cm} m{0.1cm} m{0.1cm} m{0.1cm} m{0.1cm} m{0.1cm} m{0.1cm} m{0.1cm} | c | m{0.1cm} m{0.1cm} m{0.1cm} m{0.1cm} m{0.1cm} m{0.1cm} m{0.1cm} m{0.1cm} | l }
\multicolumn{9}{c}{Normal reading} & \multicolumn{9}{c}{Displaced reading} &  \\
\cline{2-9} \cline{11-18}
1st byte & 0 & 0 & 0 & 0 & 1 & 1 & 1 & 1 & 5th byte & 0 & 0 & 0 & 0 & 1 & 1 & 1 & 1 & $\leftarrow$ Start byte \\
\cline{2-9} \cline{11-18}
2nd byte & 0 & 0 & 0 & 0 & 0 & 0 & 0 & 1 & 6th byte & 0 & 0 & 0 & 0 & 0 & 0 & 0 & 1 & $\leftarrow$ Destination \\
\cline{2-9} \cline{11-18}
3rd byte & 0 & 0 & 0 & 0 & 0 & 1 & 1 & 0 & 7th byte & 0 & 0 & 0 & 0 & 0 & 1 & 1 & 0 & $\leftarrow$ Length (First seven bit)\\
\cline{2-9} \cline{11-18}
4th byte & 1 & 1 & 1 & 1 & 0 & 0 & 0 & 0 & 8th byte & 0 & 1 & 1 & 1 & 0 & 0 & 1 & 1 & \\
\cline{2-9} \cline{11-18}
5th byte & 0 & 0 & 0 & 0 & 1 & 1 & 1 & 1 & 9th byte & 1 & 0 & 0 & 1 & 1 & 0 & 1 & 1 & \\
\cline{2-9} \cline{11-18}
6th byte & 0 & 0 & 0 & 0 & 0 & 0 & 0 & 1 & 10th byte & 1 & 0 & 0 & 0 & 0 & 1 & 0 & 0 & \\
\cline{2-9} \cline{11-18}
7th byte & 0 & 0 & 0 & 0 & 0 & 1 & 1 & 0 & 11th byte & 0 & 0 & 1 & 1 & 0 & 1 & 1 & 1 & \\
\cline{2-9} \cline{11-18}
8th byte & 0 & 1 & 1 & 1 & 0 & 0 & 1 & 1 & 12th byte & 1 & 1 & 1 & 1 & 0 & 0 & 0 & 0 & \\
\cline{2-9} \cline{11-18}
9th byte & 1 & 0 & 0 & 1 & 1 & 0 & 1 & 1 & 1st byte & 0 & 0 & 0 & 0 & 1 & 1 & 1 & 1 & \\
\cline{2-9} \cline{11-18}
10th byte & 1 & 0 & 0 & 0 & 0 & 1 & 0 & 0 & 2nd byte & 0 & 0 & 0 & 0 & 0 & 0 & 0 & 1 & \\
\cline{2-9} \cline{11-18}
11th byte & 0 & 0 & 1 & 1 & 0 & 1 & 1 & 1 & 3rd byte & 0 & 0 & 0 & 0 & 0 & 1 & 1 & 0 & \\
\cline{2-9} \cline{11-18}
12th byte & 1 & 1 & 1 & 1 & 0 & 0 & 0 & 0 & 4th byte & 1 & 1 & 1 & 1 & 0 & 0 & 0 & 0 & $\leftarrow$ End byte\\
\cline{2-9} \cline{11-18}
\end{tabular}
\caption{Example of a normal reading of a package and a displaced reading of a package. The package contains the coordinates (-8222, 515, -3699). For the displaced reading, another package equal to the first one comes after the first package.}
\label{errorPro}
\end{table}

For the example on \tableref{errorPro}, the 5th to 7th byte are equal to the start byte and the header. So if the system is looking after the start byte and find the 5th byte, before it finds the next 1st byte, it will begin to look for the header. In this case it will find it and as the 12th byte, which is in the next package, is equal to the end byte, the system will consider it have founded a complete package. But as there is also the control with the checksum, the package will probably not go through the security system. The checksum, in this case, will not be the original checksum for the package, but another part of the package. As this checksum is not calculated to fit, the chance for the check goes through is so small that it will not be considerer to happen. 

But even if the wrong package is being stopped by the control with the checksum, the 12 bytes still have been read and therefore have to be thrown. This will happen no matter if the end byte is correct or not, just as long as the system finds the start byte and the header. And in these 12 bytes, the next real start byte is also thrown away. In worst case scenario, the next three byte, after the ones that got thrown away, is equal to the start byte and header again, the same scenario will repeat it self and will go on, until the three byte no longer equals the start byte and header.

These scenarios about beginning to read a package in the middle of a package is not that common, as three byte are needed to be precisely like the start byte and the header (the last bit of the third byte is from the x coordinate, so it is not needed to be the same). The chance is small that the data and checksum parts are 8 bytes long together,  and that the system recognize the wrong start byte and header. Even if both things happens, as the coordinate will change for each package because of the moving vehicle, it will not be considered to be happening in a row. And as the system do not have a problem with one package missing, this flaw in the error handling will not considered as a problem.



The protocol of the Xbee modules are now explained and implemented, with the handling errors

