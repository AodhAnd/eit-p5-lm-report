\chapter{Digital Filter}\label{chap:digitalFilter}

When utilizing a sensor unwanted noise can arise and influence the measurements acquired. By implementing a filter it is possible to attenuate and/or enhance specific frequency components contained in the measurements. When the magnetometer, described in \secref{HardwareChoice}, is active while the vehicle is stationary, the measured angle varies approximately two degrees, see \figref{fig:StationaryMeasurementsMagnato}. The noise affecting the measurements can have a inexpedient effect on the controller which is implemented on the prototype. With ideal circumstances the magnetometer would measure an angle variation of zero degrees. Since this is not the case, implementing a filter to attenuate some of the noise could be a potential solution to get more accuracy.

There is many cons and pros when considering between implementing a analogue or digital filter. For example a analogue filter only utilizes hardware and therefore does not occupy as much of the computation time on the microprocessor as a digital filter. On the other hand since the analogue filter utillized component's it is exposed to component tolerances. Besides the microprocessor a digital filter will need no components, since the data received from the magnetometer is already digitalized.

From these few arguments it has been chosen to implement a digital filter, since the signal received from the magnetometer has already been digitalized, and it would therefore be impractical to implement an analogue filter.

In the following point form the separate sections in this chapter are illustrated. The point form has been made to make the overall steps when designing a filter more clear to the reader.

\begin{itemize}
\item Frequency analysis of measured data
\item Specific requirements
\item Filter type
\item Design
\item Implementation
\item Results
\end{itemize}

When designing and implementing a filter a few considerations needs to be made first. In the first section the measured data will be frequency analysed to give a base for creating requirements for the filter. After making specific requirements for the filter it is possible to examine and decide which filter would be suitable for fulfilling the requirements, without influencing the desired frequencies. Hereafter, it is possible to design and implement the filter, and thereafter conclude if the filter complies with what is desired.

The following section contains a frequency analysis of the measured data which is need to set up requirement for the filter.

\section{Frequency Analysis of Measured Data}
Before designing a filter it is necessary to analyse data measured with the magnetometer, to ensure the signal is not attenuate and as must noise is removed (attenuated) as possible. The data analysed is when the vehicle and the magneto is stationary, and thereby finding the stationary variations on the sensor, the acquired measurements is illustrated in \figref{fig:StationaryMeasurementsMagnato}.

\begin{figure}[H]
  \centering
 	%Trim margins @:   left        bottom       right       top
 	\adjustbox{ trim = {.15\width} {.30\height} {.15\width} {.30\height}, clip }
  {
    \includegraphics[width=1.1\textwidth]{figures/StationaryMeasurementsMagnato.pdf}
  }
  \caption{A plot, where the x-axis is time and the y-axis is the angle, of data measured with the magnetometer while the vehicle is stationary.}
  \label{fig:StationaryMeasurementsMagnato}
\end{figure}

From \figref{fig:StationaryMeasurementsMagnato} it can be seen how the angle measured vary with approximately 2 degrees. In the datasheet for the magnetometer it is explained that the sensor has a 1 to 2 degrees accuracy \cite{MagnoDatasheet}, which is consistent with the measured data. To be able to analyse measured data affected by noise, a FFT is utilized to convert the signal from time-domain to frequency domain. The FFT makes it possible to see the frequency component contained in the signal, thereby making it possible to find the signal and the noise affecting the signal. A FFT of the measured data is performed and illustrated in \figref{fig:StationaryMeasurementsMagnato}.

\begin{figure}[H]
  \centering
 	%Trim margins @:   left        bottom       right       top
 	\adjustbox{ trim = {.15\width} {.30\height} {.15\width} {.30\height}, clip }
  {
    \includegraphics[width=1.1\textwidth]{figures/FFTofStationaryMeasurements.pdf}
  }
  \caption{A FFT of the measured data illustrated in \figref{fig:StationaryMeasurementsMagnato}}
  \label{fig:FFTofStationaryMeasurements}
\end{figure}

The data measured in \figref{fig:StationaryMeasurementsMagnato} is acquired with a sampling frequency of 40 \si{Hz}. The Nyquist-Shannon Sampling Theorem says that to find the frequency components contained in a signal you need atleast twice the sampling frequency \cite{AVOppenheim}:
%
\begin{flalign}
\Omega_s &\geq 2 \cdot \Omega_N \unit{Hz}
\end{flalign}
\hspace{6mm} Where:\\
\begin{tabular}{p{1cm}lll}
& \si{\Omega_s}            	& is the sampling frequency         &\unitWh{Hz} \\
& \si{\Omega_N}				& is the Nyquist frequency			&\unitWh{Hz} \\
\end{tabular}

This is illustrated in \figref{fig:FFTofStationaryMeasurements}, where the frequency components only goes from 0 to 20 Hz on the x-axis, which is half the sampling frequency. The y-axis is the magnitude of the frequency component, measured in \si{dB}, occurring in the signal.

From the FFT performed, seen in \figref{fig:FFTofStationaryMeasurements}, a spike is present at 1 \si{Hz}, this is the DC value, i.e. the offset seen in \figref{fig:StationaryMeasurementsMagnato}. This frequency component is the signal which the filter should not attenuate. The frequency components present after 1 Hz and to 20 Hz, is the noise from the sensor, and is what is desired to filter. 

Some consideration has to be made for the filter requirements, to ensure that the filter is not influencing the system and the desired frequency needlessly.

\section{Filter Requirements} \label{sec:FilterRequirements}
It has been chosen to design the filter by using a iterative process. This is done to ensure that the filter does not take to long to execute, i.e. it will take up to much computation time on the Arduino Mega. When a digital filter is implemented it is also necessary to think about group delay, since this will change the phase of the system and cause an output delay, \todo{is there a rule of thumb for this?}. To see if the filter complies with these requirement, it is necessary to first design and implement the filter\todo{something about how much time is to much time}.

By doing so, it is found that a filter with a order of 4 is satisfactory.

Further requirements for the filter will be found by using the FFT, which is illustrated in the previous section, and by finding the pole placement of the  inner loop of the steering model, where the Magnetometer is placed, see \secref{sec:SteeringControl}. This is necessary to ensure that the poles from the filter does not affect the inner loop control. A rule of thumb is that the pole placement of the filter should be placed atleast one decade after the main pole of the system. If the filters pole placement is closer than this, it would be necessary to implement the transfer function for the filter in the steering model, which is undesirable. A bodeplot for the inner loop in the steering model, can be seen in \figref{fig:BodePlotSteeringInnerLoop}.

\begin{figure}[H]
	\centering
%	\includegraphics[scale=1]{figures/BodePlotSteeringInnerLoop.jpg}
	\caption{Frequency response for various filter types}
	\label{fig:BodePlotSteeringInnerLoop}
\end{figure}

It can be seen from the bodeplot in \figref{fig:BodePlotSteeringInnerLoop}, that the inner loops cut-off frequency is at 0.22 \si{Hz}. The cut-off frequency of the filter should therefore be at atleast 2.2 \si{Hz}. 

Since it is desired to influence the DC-signal as little as possible, see \figref{fig:FFTofStationaryMeasurements}, a low-pass filter is implemented. The ideal filter would have a passband attenuation of 0 \si{dB}, since this is not a ideal filter a maximum attenuation variation is set from 1 \si{dB} to 0 \si{dB}. The filter has been set to have a order of 4 and it is therefore possible to have a attenuation  40 \si{dB} in the stopband if the transitionband has a length of 5.831 \si{Hz}.

The sampling frequency is 30 ms, which is 33.3 \si{Hz}. The Nyquist frequency is therefore 16.67 \si{Hz}. It has been chosen to have the passband end at 8.33 \si{Hz} and the stopband to start at 14.16 \si{Hz}. This should give a cut-off frequency which is a lot higher than 2.2 \si{Hz}. Furthermore, by the use of an iterative process, it has been found that by placing the end of the passband earlier would yield a high delay in the output signal.

The filters discrete-time specification is therefore:

\textbf{Passband specification:}
\begin{flalign}
0.89125 &\leq |H(e^{j\omega}| \leq 1 \\
\end{flalign}

\textbf{Stopband specification:}
\begin{flalign}
|H(e^{j\omega})| &\leq 0.01 \\
\end{flalign}

\section{Filter Type}
The specification for the filter is set and it is possible to examine which filter would be suitable for fulfilling the specified requirements, without influencing the DC-value and the system needlessly.

\begin{figure}[H]
	\centering
	\includegraphics[scale=1]{figures/Filtertypes1.pdf}
	\caption{Frequency response for various filter types}
	\label{fig:Filtertype1}
\end{figure}

The disadvantages of a Elliptic filter is that it has both ripples in the pass- and stopband. additionally, it has a high group delay both before and at the cutoff frequency, this can be seen in \figref{fig:groupdelay}. The advantage of a Elliptic filter is that is has the sharpest cut-off frequency of the filters illustrated in \figref{fig:Filtertype1}.

The Chebyshev filter only has ripples in the passband, and a sharp cut-off frequency, but still has a high group delay before and at the cut-off frequency. The inverse Chebyshev only has ripples in the stopband, but does not have as sharp cut-off frequency as the Chebyshev and the Elliptic filter. Compared to the two former filter it has a lot less group delay before and after the cut-off frequency.

The Bessel filter has the lowest group delay before and at the cut-off frequency, see \figref{fig:groupdelay}. additionally, it does not have any ripples, neither in pass- or stopband. The disadvantage of the filter is it has the least sharpest cut-off frequency of the filters illustrated in \figref{fig:Filtertype1}.

The Butterworth filter has a sharper cut-off frequency compared to the Bessel filter and as the Bessel filter it does not have any ripples neither in the pass- or stopband. Furhtermore, the Butterworth filter has the second lowest group delay at the cut-off frequency compared to the other filters illustrated in \figref{fig:groupdelay}. 

\begin{figure}[H]
	\centering
	\includegraphics[scale=0.7]{figures/Filtertypes2.pdf}
	\caption{Frequency response illustrating group delay for various filter types}
	\label{fig:groupdelay}
\end{figure}

Because of the above-mentioned descriptions of the filter a Butterworth filter has been selected for filtering the measured data. The Butterworth does not have any ripples in the passband which could influence the DC-value, seen in \figref{fig:FFTofStationaryMeasurements}, and it upholds the requirement set for the filter in \secref{sec:FilterRequirements}. Compared to the other described filters, the Butterworth has a small group delay.

The requirements and filter type has been chosen, and it is thereby possible to design the filter which has to be implemented.

\section{Design}

\subsection{Bilinear Transform vs. Impulse Invariance transformation}
There is different methods for transferring a continuous-time filter to a discrete-time filter, the two most common transformation methods is examined. The first method which will be examined is the impulse variance transformation.

When utilizing the impulse invariance transformation a discrete filter is generated by sampling a impulse response of a continuous analogue prototype filter. The impulse response of the discrete-time filter is proportional to the impulse response of the continuous-time filter with equally spaced samples, i.e.  making the discrete-filter from a sampled continuous-time prototype filter, see \eqref{impulseresponsepropotional}, \cite{AVOppenheim}.
%
\begin{flalign}
h[nT] &= h[n] = T \cdot h(t) \big\vert_{t=nT}&
\label{impulseresponsepropotional}
\end{flalign}
%
This makes it a linear mapping, \si{\omega = \Omega \cdot T_d} for each pole in the analogue filter's transfer function on the s-plane to a pole on the z-plane. 

The main issue when utilizing the impulse variance method is that the sampling rate needs to be relatively high compared to the filter's bandwidth, to not cause aliasing \cite{LyonsR.G}. In the frequency response, seen in \figref{fig:ImpulseVariantFrequencyResponse}, it can be seen that the response of a 6th-order Butterworth filter transformed by impulse invariance, has a frequency response which is larger than from 0 to \si{pi}, hence causing aliasing.

\begin{figure}[H]
	\centering
	\includegraphics[scale=0.2]{figures/BilinearFrequencyResponse.pdf}
	\caption{A frequency response of a transformed impulse variance 6th order Butterworth filter \cite{AVOppenheim}.}
	\label{fig:ImpulseVariantFrequencyResponse}
\end{figure}

The Bilinear transform also utilize a analogue prototype filter to convert from s-domain to z-domain, when designing a digital filter. This transformation type is not a linear transformation from the s-domain to the z-domain as it is with the impulse variance method. Instead it is a non-linear transformation which is mapping poles from \si{-\infty < \Omega < \infty} in the s-domain to \si{-\pi < \omega < \pi} in the z-domain \cite{OlesSlides}. This yields a relationship between the continuous-time frequencies to the discrete-time frequencies called pre-warping.
%
\begin{flalign}
\omega &= 2 \cdot \arctan(\frac{\Omega \cdot T_d}{2}) \rightarrow \Omega = \frac{2}{T_d} \cdot \tan(\frac{\omega}{2}) &
\label{eq:bilinearprewarp}
\end{flalign}
%
A frequency response of the same filter, as the frequency response for the impulse invariance, is illustrated in \figref{fig:BilinearFrequencyResponse}. The bilinear transform avoids aliasing problems because it maps the s-planes imaginary axis onto the z-planes unit circle \cite{AVOppenheim}, which can be seen in \figref{fig:S-planeVsZ-plane}.
%
\begin{figure}[H]
	\centering
%	\includegraphics[scale=0.3]{figures/SplaneVsZplane.pdf}
	\caption{s-domain and z-domain \cite{AVOppenheim}.}
	\label{fig:S-planeVsZ-plane}
\end{figure}
%
From \figref{fig:BilinearFrequencyResponse} it can be seen that the frequency response does not exceed the limit from 0 to \si{pi}, hence not causing a aliasing problem \cite{AVOppenheim}.

\begin{figure}[H]
	\centering
	\includegraphics[scale=0.2]{figures/ImpulseVariantFrequencyResponse.pdf}
	\caption{A frequency response of a transformed bilinear transform 6th order Butterworth filter \cite{AVOppenheim}.}
	\label{fig:BilinearFrequencyResponse}
\end{figure}

It has been chosen to use the bilinear transformation when designing the filter. This is mainly do to the issue with aliasing that occurs when utilizing impulse variance transform which does not happen when utilizing bilinear transform.

\subsection{Designing a low-pass Butterworth filter using the bilinear transform}
When utilizing bilinear transform it is necessary to frequency warp the passband frequency and the stopband frequency, from \secref{sec:FilterRequirements}. This is done by utilizing the relationship between the discrete-time frequencies in the z-domain to the continuous-time frequencies in the s-domain illustrated in \eqref{eq:bilinearprewarp}. 
%
\begin{flalign}
\Omega_p &= 80 \unit{Hz}\\
\Omega_s &= 333.2 \unit{Hz}
\end{flalign}
\hspace{6mm} Where:\\
\begin{tabular}{p{1cm}lll}
& \si{\Omega_p} & is the passband frequency in continuous time &\unitWh{Hz} \\
& \si{\Omega_s}	& is the stopband frequency in continuous time &\unitWh{Hz} \\
\end{tabular}

By utilizing the magnitude squared function for a Butterworth filter, it is possible to find the cut-off frequency and the order, N, of the filter:
%
\begin{flalign}
\eq{|H(e^{j\omega}|^2}{\frac{1}{1+(\frac{\Omega}{\Omega_c})^{2N}}}
\end{flalign}
\hspace{6mm} Where:\\
\begin{tabular}{p{1cm}lll}
& \si{\Omega}       & is the pre-warped frequencies  &\unitWh{Hz} \\
& \si{\Omega_C}		& is the cut-off frequency &\unitWh{Hz} \\
& \si{|H(e^{j\omega}|} & is the attenuation requirements set for the different bands from \secref{sec:FilterRequirements} &\unitWh{dB}
\end{tabular}

Since there is two unknown variables, the cut-off frequency and the order N, it is possible to calculate the values of these by utilizing the two equations with two unknowns:
%
\begin{flalign}
0.89125^2 &= \frac{1}{1+(\frac{80}{\Omega_C})^{2N}} \quad \wedge \quad 0.01^2 = \frac{1}{1+(\frac{333.2}{\Omega_C})^{2N}}
\end{flalign}
%
The cut-off frequency is calculated to 96.02 \si{Hz} and the order, N, of the filter to 3.7. The order is rounded to the nearest integer, i.e \si{4}. When the order is known it is possible to utilize the following equation for finding the poles for a Butterworth filter.
%
\begin{flalign}
\eq{P_k}{\Omega_c \cdot e^{j(\frac{2 \cdot k -1}{2 \cdot N} \cdot \pi + \frac{\pi}{2})}}
\end{flalign}
%
Where k equal 1,2 \si{\dotsc N}.

Since it would desirable to have a stable system, the poles should only be located on the left side of the y-axis in the s-plane.
%
\begin{flalign}
\eq{P_1}{\Omega_c \cdot e^{j \cdot \frac{5}{8} \pi}} \\
\eq{P_2}{\Omega_c \cdot e^{j \cdot \frac{7}{8} \pi}} \\
\eq{P_3}{\Omega_c \cdot e^{j \cdot \frac{9}{8} \pi}} \\
\eq{P_4}{\Omega_c \cdot e^{j \cdot \frac{11}{8} \pi}}
\end{flalign}
%
The general transfer function for a Butterworth filter is defined as:
%
\begin{flalign}
\eq{H(s)}{\frac{G_o}{\prod\limits_{k = 1}^N (s-P_k)}}
\end{flalign}
\hspace{6mm} Where:\\
\begin{tabular}{p{1cm}lll}
& \si{G_o}       & is the gain of the filter  &\unitWh{\cdot} \\
\end{tabular}

\todo{something about gain}

\begin{flalign}
\eq{H(s)}{\frac{\Omega_c^4}{(s-\Omega_ce^{j\cdot \frac{5}{8} \cdot \pi})(s-\Omega_ce^{j\cdot \frac{7}{8} \cdot \pi})(s-\Omega_ce^{j\cdot \frac{9}{8} \cdot \pi})(s-\Omega_ce^{j\cdot \frac{11}{8} \cdot \pi})}}
\end{flalign}

It can be seen from the placement of the poles, that there is two complex pole-pair. By utilizing the rule \si{s = s*}, it is possible to change the transfer function.
%
\begin{flalign}
\eq{H(s)}{\frac{\Omega_c^4}{(s-\Omega_ce^{j\cdot \frac{5}{8} \cdot \pi})(s-\Omega_ce^{j\cdot \frac{7}{8} \cdot \pi})(s-\Omega_ce^{-j\cdot \frac{7}{8} \cdot \pi})(s-\Omega_ce^{-j\cdot \frac{5}{8} \cdot \pi})}}
\end{flalign}
%
Transforming the complex poles to standard form:
%
\begin{flalign}
\eq{H(s)}{\frac{\Omega_c^4}{(s^2 + 2 \cdot \cos{(\frac{7}{8} \cdot \pi)} \cdot \Omega_c \cdot s + \Omega_c^2)(s^2 + 2 \cdot \cos{(\frac{5}{8} \cdot \pi)} \cdot \Omega_c \cdot s + \Omega_c^2)}}
\end{flalign}

\todo{Bodeplot}

The next step would be to transfer the continuous-time Butterworth filter to the z-domain.

\subsection{Transforming the filter to Z-domain}

\begin{flalign}
\eq{s}{\frac{2}{T_d}(\frac{1-z^{-1}}{1+z^{-1}})}
\end{flalign}

Standard formula:

\begin{flalign}
H(z) &= \frac{B(z)}{A(z)} = \frac{b_0 + b_1z^-1 + b_2z^-2 + \dotsc + b_Nz^{-N}}{1 + a_1z^-1 + a_2z^-2 + \dotsc + a_Mz^{-M}}
\end{flalign}

\begin{flalign}
H(z) &= \frac{B(z)}{A(z)} = \frac{0.1326 + 0.5305z^{-1} + 0.7957z^{-2} + 0.5305z^{-3} + 0.1326z^{-4}}{1 + 0.4515z^{-1} + 0.5505z^{-2} + 0.09825z^{-3} + 0.02167z^{-4}}
\end{flalign}

\section{Implementation}


\lstset{language=Matlab, caption={Code Implementation in Matlab}, label=lst:FilterMatlabImplementation}
\begin{lstlisting}
for i = 1:707

  BUF0 = input(i) - (a1*BUF1 + a2*BUF2 + a3*BUF3 + a4*BUF4)
	
  output(i) = BUF0*b0 + b1*BUF1 + b2*BUF2 + b3*BUF3 + b4*BUF4
    	
  BUF4 = BUF3
  BUF3 = BUF2
  BUF2 = BUF1
  BUF1 = BUF0
    
 end
\end{lstlisting}

\section{Results}