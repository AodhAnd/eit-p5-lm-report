\chapter{Filter}
The vehicle 

\section{Requirements}
Before selecting which filter to design and implement, it is necessary to examine the requirements needed for performing the necessary filtering

\textbf{Passband specification:}
\begin{flalign}
0.89125 &\leq |H(e^{j\omega}| \leq 1 \unit{dB}\\
0 &\leq \omega \leq 0.3\pi \unit{rad}
\end{flalign}

\textbf{Stopband specification:}
\begin{flalign}
|H(e^{j\omega})| &\leq 0.01 \unit{dB}\\
0.3 &\leq \omega \leq \pi \unit{rad}
\end{flalign}

\subsection{Filter Type}
The specification of the filter is set and it is thereby possible to examine which filter would be suitable for fulling the specified requirements, without influencing the desired frequencies needlessly.

\begin{figure}[H]
	\centering
	\includegraphics[scale=1]{figures/Filtertypes1.pdf}
	\caption{Frequency response for various filter types}
	\label{Filtertype1}
\end{figure}

\begin{figure}[H]
	\centering
	\includegraphics[scale=0.7]{figures/Filtertypes2.pdf}
	\caption{Frequency response illustrating group delay for various filter types}
	\label{Filtertype1}
\end{figure}

Because of the above-mentioned arguments a Butterworth filter has been selected for filtering the measured data.

\section{Design}

\subsection{General transfer function}

Pre-warping: 

\begin{flalign}
\eq{\Omega}{2 \cdot T_d \cdot \tan{\frac{0.3 \pi}{2}}} \\
\end{flalign}

Magnitude squared function:

\begin{flalign}
\eq{|H(e^{j\omega}|^2}{\frac{1}{1+(\frac{\Omega}{\Omega_c})^{2N}}} \\
\end{flalign}

Poles:

\begin{flalign}
\eq{P_k}{\Omega_c \cdot e^{j(\frac{2 \cdot k -1}{2 \cdot N} \cdot \pi + \frac{\pi}{2})}} \\
\eq{k}{1,2 \dotsc N}
\end{flalign}

General transfer function:

\begin{flalign}
\eq{H(s)}{\frac{G_o}{\prod\limits_{k = 1}^N (s-P_k)}}
\end{flalign}

Transfer function:

\begin{flalign}
\eq{H(s)}{\frac{G_o}{(s-e^{j\cdot \frac{2}{3} \cdot \pi})}}
\end{flalign}

The next step would be to transfer the continuous-time Butterworth filter to the z-domain.

\subsection{Bilinear Transform vs. Impulse Invariance transformation}
Before transferring a continuous-time filter to the z-domain, the two most common transformation methods is examined.

\textbf{Bilinear transform} is 

\begin{figure}[H]
	\setcounter{subfigure}{0}
	\centering
	\begin{subfigure}{.45\textwidth}
		\centering
		\includegraphics[width=\linewidth]{figures/BilinearFrequencyResponse.pdf}
		\caption{A frequency response of a transformed impulse variance 6th order Butterworth filter}
		\label{fig:ImpulseVarianceResponse}
	\end{subfigure}
	\hfill
	\begin{subfigure}{.45\textwidth}
		\centering
		\includegraphics[width=\linewidth]{figures/ImpulseVariantFrequencyResponse.pdf}
		\caption{A frequency response of a transformed bilinear transform 6th order Butterworth filter}
		\label{fig:BilinearTransformResponse}
	\end{subfigure}
	\caption{frequency response of a 6th order Butterworth filter, transformed from continuous time to the z-domain by two different methods named Bilinear transform and Impulse Variance}
		\label{fig:bilinearandimpulsevariance}
\end{figure}

\subsection{Transforming the filter to Z-domain}


Standard formula:

\begin{flalign}
H(z) &= \frac{B(z)}{A(z)} = \frac{b_0 + b_1z^-1 + b_2z^-2 + \dotsc + b_Nz^{-N}}{1 + a_1z^-1 + a_2z^-2 + \dotsc + a_Mz^{-M}}
\end{flalign}

\section{Implementation}

\subsubsection{Direct Form I}

\subsubsection{Direct Form II}

\section{Results}