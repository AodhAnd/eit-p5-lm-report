\section{Hall-sensor}

The velocity calculation is done trough the Hall sensors. There are four magnets on the gear at approximatly a quarter of a turn of distance each. One way of getting the speed is by calculating the time between each outputs and the distance traveled during that time. A plot of the speed can be seen on \figref{unfilteredHall}.

\begin{figure}[H]
	\centering
	\includegraphics[scale=0.9]{figures/unfilteredHall.pdf}
	\caption{Plot of an unfiltered measurement by the Hall Sensors at full speed}
	\label{unfilteredHall}
\end{figure}


This is a plot of the speed, with unaccurate measurement because of the uneven placement of the magnets, with a calculation of the time between each magnets.\\


The new approach is to get the time the wheel take to make a full turn, to have the exact time and distance of a rotation. The speed will be calculated from a full turn every magnets, compared to the last time it was registered, four outputs before. A plot of the measurements can be seen on \figref{filteredHall}.

\begin{figure}[H]
	\centering
	\includegraphics[scale=0.9]{figures/filteredHall.pdf}
	\caption{Plot of an filtered measurement by the Hall Sensors at full speed}
	\label{filteredHall}
\end{figure}

The difference between the two plots is that at constant speed, the time measured at each outputs is uneven on \figref{unfilteredHall}, and even in \figref{filteredHall}. A flowchart of the implementation of the functions of the Hall Sensors can be seen on \figref{hallFlowchart}.

\begin{figure}[H]
	\centering
	\includegraphics[scale=0.9]{figures/hallFlowchart.pdf}
	\caption{Flowchart of the two main functions \textit{tSpeed} and \textit{getSpeed} of the Hall Sensors implementation}
	\label{hallFlowchart}
\end{figure}

This flowchart explains the way of getting the time at each outputs. The Hall Sensors use the timers 4 and 5 to register the time of the output. The first function \textit{tSpeed} describes the storing of the two timers values in the registers, the second is a subfunction of the first one, and the last function \textit{getSpeed} is the calculation of the speed from the raw data of a variable in the function \textit{tSpeed}.\\


\subsubsection{Minimum speed because of the registers}

The Hall Sensors have a minimum speed because of the maximum holding time in the buffer, and maximum speed because of the minimu sampling time of the sensor. The timers can only count up to 65 535 µs, and the sensor can not measure a time less than 64 µs. \\
A quarter of turn of the gear is 166 mm long, so the minimum and maximum speed are 0.25 m/s and 260 m/s.	



\subsubsection{FIR Filter}

Using the four magnets of the wheel and recording the time values through the timers involve using a  FIR Filter(Finite Impulse Response Filter). An overview of the filter can be seen on \figref{FIRFilter}.


\begin{figure}[H]
	\centering
	%\includegraphics[scale=0.9]{figures/FIRFilter.pdf}
	\caption{Overview of an FIR Filter}
	\label{FIRFilter}
\end{figure}