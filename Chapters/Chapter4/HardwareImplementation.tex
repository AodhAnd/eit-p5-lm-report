\section{Hardware Implementation}
\subsection{Microcontroller peripherals}
Before choosing which pins on the Arduino Mega to connect the various hardware to, the peripherals\footnote{Hardware modules, such as timers, UARTs, etc.} in the microcontroller needs to be evaluated (each peripheral is only available on certain pins). First, the timers are considered. The microcontroller has 6 hardware timers, \emph{Timer 0 - Timer 5}.

Timer 0 is reserved by the Arduino itself, and not be touched. \todo{source} This leaves 5 timers, that can be used freely.

It would be advantageous to utilize the input capture\footnote{An input capture timer samples a free running counter, whenever triggered by  an external event} functionality of the timers to read the hall sensors. After examining the Arduino Mega pinout \todo{source}, it is seen that only Timer 4 and 5 have their input capture pins available on the board. These are therefore chosen for the hall sensors.
   
The realtime operating system needs a timer, for running it's scheduler. Timer 1 is chosen for this, for no other reason than it being the default setting.

This leaves Timer 2 and 3. According to the datasheet of the microcontroller, Timer 2 is an 8 bit timer, and Timer 3 is 16 bit. 

Timer 3 is chosen for the servo. This, because the higher resolution makes it easier to generate the short duty cycles required by the servo (500us to 2500us on-time, in a 30ms period, see \subsecref{Servo})

The last timer is chosen for the DC-motor.

As the timers has now been designated, the next will be the various serial ports. The XBee module needs to connect to a serial port, and UART3 is chosen, as it makes it easiest to route the PCB traces. The SD-card needs an SPI connection, and must therefore be connected to the SPI pins. Lastly, the "9 Degrees of Freedom"-board needs an $\text{I}^2\text{C}$ connection, and has to be connected to the Arduino's $\text{I}^2\text{C}$ pins.

As all hardware modules have now been assigned pins on the Arduino, the electrical requirements of each module will now be considered.

\subsection{Electrical considerations}

Before connecting hardware modules to the Arduino pins, signal voltage levels need to be considered:

\begin{itemize}
\item The Arduino itself uses 5V logic levels.
\todo{source} 
\item The "9 Degrees of Freedom"-board needs 5V supply voltage and $\SI{3,3}{V}$ logic levels
\todo{source}
\item The servo needs $\SI{4,8}{V}$ - 6V supply voltage and 5V signal levels
\todo{source}
\item The SD-card needs $\SI{3,3}{V}$ supply voltage and signal levels
\todo{source}
\item The Hall sensors need $\SI{3,5}{V}$ - 24V supply, and a pull up resistor to define the logic level.
\todo{source}
\item The XBee needs $\SI{3,3}{V}$ supply voltage and signal levels
\todo{source}
\end{itemize} 

The Arduino has regulated 5V and $\SI{3,3}{V}$ supply rails available, which will be used to power the respectable hardware modules. As the servo can potentially draw a lot of current, and overload the Arduino, it will be connected to a dedicated 5V voltage regulator, powered directly from the battery.

A low-dropout type is chosen, to make sure it works, even at low battery voltages. \todo{ref?}

All one-directional connections from the Arduino to  $\SI{3,3}{V}$ logic level inputs will be connected to a CD4050 voltage level translator. \todo{ref}

One-directional connection from  $\SI{3,3}{V}$ logic level output to Arduino inputs will be connected directly, as $\SI{3,3}{V}$ is above the minimum high level threshold. \todo{source}

As the $\text{I}^2\text{C}$ bus is using bi-directional connections, and needs to connect a $\SI{3,3}{V}$ system to a 5V system, this needs to be addressed as well. The creators of $\text{I}^2\text{C}$, Philips, recommends doing this with two mosfets, see \todo{ref}.
=======
This section will be about the implementation of the hardware components, that will be used on the vehicle, which was pick in \secref{Hardwarechoice}. The main component is the microcontroller, the Arduino Mega 2560, where all the other components is connected to. 