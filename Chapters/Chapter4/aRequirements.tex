\chapter{Prototype Requirements} \label{Requirements}

\begin{enumerate}
%\item \textbf{It shall be possible to track the vehicle's position}
%	\begin{itemize}
%	\item[] why and from where?
%	\end{itemize}
\item \textbf{It shall be possible for the vehicle to receive it own location wirelessly from the GoT system, through a computer.}
	\begin{itemize}
	\item[] The vehicle needs to be able to receive its own location, additionally, it should be wireless so the vehicle can drive without cables attached, and thereby be able to move freely, \secref{Finalprototype}
	\end{itemize}
\item \textbf{It shall be possible for the prototype to disregard incorrect packets transmitted from the computer}
	\begin{itemize}
	\item[] To ensure that invalid data is not utilized, \secref{Finalprototype}
	\end{itemize}
	\item \textbf{The prototype must be able to disregard erroneous coordinates sent from the GoT system}
	\begin{itemize}
	\item[] If the GoT system delivers an unrealistic coordinate, e.g. created by noise, to the vehicle, \secref{Finalprototype}.
	\end{itemize}
\item \textbf{The prototype must be able to access the route, which it has to follow, from a storage space located on the vehicle}
	\begin{itemize}
	\item[] To simulate the use-case, see \secref{sec:UseCase}, the route needs to be stored on the vehicle, and thereby making it accessible for the functionality course correction, \secref{Finalprototype}.
	\end{itemize}
\item \textbf{The prototype must be able to shut down, if the battery voltage is below its cut-off specification}
	\begin{itemize}
	\item[] To ensure the battery is not damaged, \secref{Finalprototype}.
	\end{itemize}
\item \textbf{It shall be possible for the prototype to follow a predetermined route}
	\begin{itemize}
	\item[] This is to simulate a lawn mower, which is able to cut grass by following a route, \secref{sec:PrototypeConstraints}
	\end{itemize}
\item \textbf{It shall be possible for the prototype to return to the predetermined route if disturbed}
	\begin{itemize}
	\item[] E.g. if the vehicle slips or is moved, \secref{sec:UseCase}
	\end{itemize}
\item \textbf{The prototype shall be able to keep a predetermined velocity when going up - or downhill and when turning}
	\begin{itemize}
	\item[] To ensure an evenly cut of the grass, \secref{sec:UseCase}
	\end{itemize}
\end{enumerate}
