\section{Magnetometer}\label{sec:magnetoSensor}
The ``9 degrees of freedom'' sensor board includes three sensors, of which one is the magnetometer chip HMC5883L \cite{HMC5883L}. Since the Earth's magnetic field only moves slightly, it is possibe to neglect this movement from the sensor's point of view. The magnetometer uses this property to calculate the three space components (x,y,z) of the Earth's magnetic field magnitude, therefore indicating the North pole's position relative to the sensor's own orientation.\\
The coordinates are sent through \si{I^2C} each time asked by the software onboard the Arduino. This means that it is possible to set the sampling time according to the needs of the steering controller part, within the sensor's sampling limits. 
By maintaining the sensor in a horizontal position and using trigonometry it is possible to get an angle equivalent to a compass heading, with the North as a reference of \si{0^{\circ}} :
\begin{flalign}
\eq{\theta_{heading}}{\frac{180}{\pi} \cdot \arctan\left(-\frac{y}{x}\right)} \unit{^{\circ}}
\label{eq:headingTrigonometry}
\end{flalign}
\hspace{6mm} Where:\\
\begin{tabular}{p{1cm}lll}
& \si{\theta_{heading}} & is the sensor's heading                  		&\unitWh{^{\circ}}\\
& \si{\arctan} 			& is the reciprocal of the \si{\tan} function    &\unitWh{rad}\\
& \si{y} 				& is the y-component of the measured magnitude 	&\unitWh{G}\\
& \si{x} 			    & is the x-component of the measured magnitude 	&\unitWh{G}\\
\end{tabular}

In \eqref{eq:headingTrigonometry}, the angle calculated from the arctangente function is converted from radians to degrees to facilitate the work with angle orientations later in the code.
%
However, to be able to use the heading values gotten from the sensor and this formula, it is necessary to first calibrate the sensor. This allows the sensor to account for magnetic disturbances that could happen in its environment, see \appref{app:magnetoCalibration}. It might possible though, that unpredictable disturbances occur during the mowing of the lawn. This is why a digital filtering approach is considered in \chapref{chap:digitalFilter}.

With these sensors up and running, it is possible to put up models for the different subsystems of the vehicle as well as measure parameters to put into the models simulations and verify them.