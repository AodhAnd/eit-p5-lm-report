\chapter{Discussion}
Throughout this project a prototype has been designed on the given technology base. The base, being a tracked vehicle is not the most obvious choice for general purpose lawn mowing. However the purpose of the project was to construct a functional prototype to demonstrate the idea of using a local positioning system, GoT, to control the vehicle's path on a lawn. This was achievable with the platform at hand, however the control designed for the tracked vehicle, while having common points with control of wheeled vehicles, is not as portable if an end product was to be designed.\\
The idea of using the GoT system for navigation has been investigated and even though the combined system test was not carried out, due to disturbances in the magnetic field in the Vicon room, the subsequent system tests were largely successful, and proved the concept. To provide further proof of concept however, it would be preferable to test the combined system outdoors. Furthermore, an enhancement on the positioning system could improve the route following by gaining in accuracy. \\
An other thing is to consider the route planning itself, which has not been done in its entirety. As of now, there is a route control system, that handle the execution of the route following. An algorithm to plan the path with a means of initializing the edges of the lawn, would be interesting additions to the current design. Here it was discussed that the GoT transmitter, placed on the vehicle, could be removable. This would allow the user to take the transmitter off and through a simple interface, e.g. holding down a button, could activate recording of coordinates and by walking along the edges of the lawn, be able to initialize the area in which the lawn mover had to move.\\
The prototype could also be taken further by implementation on an actual lawn mower, including control of the blade, concerning grass height. The speed of the vehicle might also need to be altered if the grass has different densities or length, in order to allow an evenly cut lawn.\\
Several other small features could be added, like detection of humidity to make sure that the lawn is only cut when it is dry enough that a good result is obtained. An obstacle detection functionality could also be added to account for unmapped objects on the lawn that could be in the lawn mower's path, e.g. stones, child toys or pets. The idea of a grass length detector was also suggested. This would allow to adapt the speed of the lawn mower according to the grass length, still to ensure that it is evenly cut.

Furthermore, the distance controller, and therefore the whole steering controller, did not get tested, as the magnetometer did not work properly inside the Vicon room. This made the two requirements for the steering untestable. By taking the GoT system outside, where the magnetometer works, and calibrating the system to the new location, a test could be made. This would be a sufficient test since, of course, a lawn mover operates outdoors. However, the time limitation of the project resulted in the test being down-prioritized, due to the somewhat time consuming setup and calibration of the GoT system.

The last requirement, that has not been fulfilled, was the requirement about nonvolatile storage on board the vehicle. It was attempted to include an SD-card, and while it was possible to read and write to/from the SD-card, it was not made to function within the real time operating system. This problem can undoubtedly be solved, however the time limitation and other priorities resulted in constraining the prototype from this requirement. This functionality was not prioritized since it did not have any significant impact on the resulting state of the prototype.














%Needed things in this discussion:

%PART 1
%Sensor problem with the GoT and the magnetometer (Outdoor)
%Storage
%Filter

%PART 2
%Other vehicle
%Further dectors
%Route planning
%User interface