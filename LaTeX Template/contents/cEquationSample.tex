\section{Equation Sample}

Ohms Law:
\begin{flalign}
  U &= I \times R \unit{\volt}
  \label{eq1}
\end{flalign}
%
Some explanation:
\begin{flalign}
  [Equation] &= [Number] \unit{Unit}
  \label{eq2}
\end{flalign}
%
Some explanation:
\begin{flalign}
  [Equation] &= [Number] \unit{Unit}
  \label{eq3}
\end{flalign}
%
Some explanation:
\begin{flalign}
  [Equation] &= [Number] \unit{Unit}
  \label{eq4}
\end{flalign}
%
Some explanation:
\begin{flalign}
  [Short Equation] &= [Number] \unit{Unit}
  \label{eq5}\\ %<-------------------------------------------------| Remember linebreak AFTER
  [Somewhat Longer Equation] &= [Number] \unit{Unit} %             | label when writing multiple
  \label{eq6}\\ %<-------------------------------------------------| equations.
  [Somewhat Quite a Lot Longer Equation] &= [Number] \unit{Unit}
  \label{eq7}
\end{flalign}
%
%
\eqref{eq1}\\
%
\eqrefTwo{eq1}{eq2}\\
%
\eqrefThree{eq1}{eq2}{eq3}\\
%
\eqrefFour{eq1}{eq2}{eq3}{eq4}\\
%
\eqrefFive{eq1}{eq2}{eq3}{eq4}{eq5}\\
%
\eqrefSix{eq1}{eq2}{eq3}{eq4}{eq5}{eq6}\\
%
\eqrefSeven{eq1}{eq2}{eq3}{eq4}{eq5}{eq6}{eq7}\\
%
\Eqref{eq1}\\
%
\EqrefTwo{eq1}{eq2}\\
%
\EqrefThree{eq1}{eq2}{eq3}\\
%
\EqrefFour{eq1}{eq2}{eq3}{eq4}\\
%
\EqrefFive{eq1}{eq2}{eq3}{eq4}{eq5}\\
%
\EqrefSix{eq1}{eq2}{eq3}{eq4}{eq5}{eq6}\\
%
\EqrefSeven{eq1}{eq2}{eq3}{eq4}{eq5}{eq6}{eq7}
%
\pagebreak