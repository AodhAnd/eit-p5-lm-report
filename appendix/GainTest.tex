\pagebreak
\section{Gain and time-constant of the vehicle} %\label{put a label here and uncomment}
\textbf{Name: Group 510}\\
\textbf{Date: 13/11 - 2015}

\subsection{Purpose}
The purpose of this test is to find the vehicle's gain.

\subsection{Theory}

\subsection{Setup}

\subsection{List of Equipment}

\begin{table}[H]
\begin{tabular}{|p{10cm}|p{4cm}|}
\hline%------------------------------------------------------------------------------------
  \textbf{Instrument}                     &  \textbf{Type}       \\
\hline%------------------------------------------------------------------------------------
  Computer                                &  HP 8460P    \\
\hline %-----------------------------------------------------------------------------------
\end{tabular}
\end{table}

\subsection{Procedure}

\begin{enumerate}
  \item A
  \item B
  \item C
\end{enumerate}

\subsection{Results}

\begin{figure}[H]
  \centering
 	%Trim margins @:   left        bottom       right       top
 	\adjustbox{ trim = {.15\width} {.30\height} {.15\width} {.30\height}, clip }
  {
    \includegraphics[width=1.2\textwidth]{figures/SpeedStepGainTest.pdf}
  }
  \caption{A plot of a measured armature resistance, with a red line indicating the an average value.}
  \label{armatureResistance}
\end{figure}

\begin{figure}[H]
  \centering
 	%Trim margins @:   left        bottom       right       top
 	\adjustbox{ trim = {.15\width} {.30\height} {.15\width} {.30\height}, clip }
  {
    \includegraphics[width=1.2\textwidth]{figures/GainOfEachSpeedStep.pdf}
  }
  \caption{A plot of a measured armature resistance, with a red line indicating the an average value.}
  \label{armatureResistance}
\end{figure}



