\pagebreak
\subsection{Armature Inductance}%\label{put a label here and uncomment}
\textbf{Name: Group 510}\\
\textbf{Date: 30/09 - 2015}

\subsubsection{Purpose}
The purpose of the test is to measure the Armature inductance $L_a$.

\subsubsection{Setup}
\begin{figure}[H]
  \centering
	\includegraphics[scale=0.5]{figures/MotorTest2.pdf}
	\caption{Setup diagram}
\end{figure}

\subsubsection{List of Equipment}

\begin{table}[H]
\begin{tabular}{|l|l|p{4cm}|}
\hline%--------------------------------------------------------------------------------
  \textbf{Instrument}                    &  \textbf{AAU-no.}  &  \textbf{Type}       \\
\hline%--------------------------------------------------------------------------------
  Power Supply ($0 - 32$ V) ($0 - 10$ A) &  77076             &  Ea - ps 7032 - 100  \\
\hline%--------------------------------------------------------------------------------
  AC/DC Current Clamp (Output: 100 mV/A) &  78550             &  FLUKE i30s          \\
\hline%--------------------------------------------------------------------------------
  Oscilloscope                           &  64672             &  Agilent DSO6034A    \\
\hline%--------------------------------------------------------------------------------
  Clamp for fixing the motor             &  03039             &                      \\
\hline%--------------------------------------------------------------------------------
\end{tabular}
\end{table}

\subsubsection{Procedure}

\begin{enumerate}
  \item Fix the motor shaft so it does not turn.
  \item Start with the power supply disconnected and turn on the oscilloscope.
  \item On the oscilloscope press the "mode"-key choose the "normal"-option and push the "single"-key.
  \item To prevent false triggering on the oscilloscope set the trigger value to %\fxnote{input value from extracted data} mV with the turn-key.
  \item To give the motor a pulse of $5$ V, put the power supply to $5$ V and connect it.
  \item Insert a USB-pin in the oscilloscope and press the save key to extract the data.
\end{enumerate}

\subsubsection{Results}

\begin{figure}[H]
	\centering
	\includegraphics[scale=.4]{figures/Exercise2}
	\caption{Plot of test results}
\end{figure}

The $L_a$ inductor is calculated from the time constant $\tau = \frac{L_a}{R_a}$. $R_a$ is know from Test \ref{}, which is $0.178 \Omega$. $\tau$ is measured, when the current is at 63.2\% of the value at steady state. $\tau$ from this test is $0.67 * 10^{-3} s$, therefore is $L_a = 119.26 \mu H$.
