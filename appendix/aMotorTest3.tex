\pagebreak
\subsection{Tachometer Constant} %\label{put a label here and uncomment}
\textbf{Name: Group 510}\\
\textbf{Date: 30/09 - 2015}

\subsubsection{Purpose}
The superpose of the test is to measure verify that tachometer constant (in V) is 0.030 multiplied by the motor velocity in radians per second.

\subsubsection{Setup}
\begin{figure}[H]
  \centering
	\includegraphics[scale=0.5]{figures/MotorTest3.pdf}
	\caption{Use-Case Diagram}
	\flushleft
\end{figure}

\subsubsection{List of Equipment}

\begin{table}[H]
\begin{tabular}{|l|l|p{4cm}|}
\hline%--------------------------------------------------------------------------------
  \textbf{Instrument}                    &  \textbf{AAU-no.}  &  \textbf{Type}       \\
\hline%--------------------------------------------------------------------------------
  Power Supply ($0 - 32$ V) ($0 - 10$ A) &  77076             &  Ea - ps 7032 - 100  \\
\hline%--------------------------------------------------------------------------------
  Multimeter                             &  60764             &  Fluke 189 True RMS  \\
\hline%--------------------------------------------------------------------------------
  Optical tachometer                     &  08246             &  Shimpo DT-205       \\
\hline%--------------------------------------------------------------------------------
\end{tabular}
\end{table}

\subsubsection{Procedure}

\begin{enumerate}
  \item Adjust voltage of power supply till you reach 6 V on the multimeter over the tachometer.
  \item Measure the RPM with the Optical tachometer.
\end{enumerate}

\subsubsection{Results}
Measured: 1933 RPM
We use the measured RPM to verity a tachometer constant of 0.03:
\begin{flalign}
  \frac{1933}{60} \cdot 2 \cdot \pi \cdot 0.03 &= 6.07 \approx 6 \unit{V}
  \label{eqTachometerConstant}
\end{flalign}