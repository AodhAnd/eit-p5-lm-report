\pagebreak
\subsection{Motor Constant} %\label{put a label here and uncomment}
\textbf{Name: Group 510}\\
\textbf{Date: 30/09 - 2015}

\subsubsection{Purpose}
The purpose of this of this test is to measure the motor constant $K_t$.

\subsubsection{Setup}
\begin{figure}[H]
  \centering
	\includegraphics[scale=0.5]{figures/MotorTest5.pdf}
	\caption{Setup diagram}
\end{figure}

\subsubsection{List of Equipment}

\begin{table}[H]
\begin{tabular}{|l|l|p{4cm}|}
\hline%------------------------------------------------------------------------------------
  \textbf{Instrument}                        &  \textbf{AAU-no.}  &  \textbf{Type}       \\
\hline%------------------------------------------------------------------------------------
  Multimeter 1                               &  60764             &  Fluke 189 True RMS  \\
\hline%------------------------------------------------------------------------------------
  Multimeter 2                   		         &  60769             &  Fluke 189 True RMS  \\
\hline%------------------------------------------------------------------------------------
  Power Supply ($0 - 32$ V) ($0 - 10$ A)     &  77076             &  Ea - ps 7032 - 100  \\
\hline%------------------------------------------------------------------------------------
  Torque sensor                              &  08772             &  Icom                \\
\hline%------------------------------------------------------------------------------------
\end{tabular}
\end{table}

\subsubsection{Procedure}

\begin{enumerate}
  \item Fix the motor shaft so it does not turn.
  \item Turn on the two multimeter in current and voltage mode respectively.
  \item Start by setting the power supply at 1 A current limiting.
  \item Turn on the supply and note the voltage across the torque sensor.
  \item Repeat the previous two steps up to 10 A with 1 A increments.
\end{enumerate}

\subsubsection{Results}

\begin{table}[H]
\begin{tabular}{|l|l|l|}
\hline%----------------------------------------
  \textbf{Input (A)}  & \textbf{Output (V)}  \\
\hline%----------------------------------------
  $1$                 &            $0.107$    \\
\hline%----------------------------------------
  $2$                 &            $0.113$    \\
\hline%----------------------------------------
  $3$                 &            $0.118$    \\
\hline%----------------------------------------
  $4$                 &            $0.122$    \\
\hline%----------------------------------------
  $5$                 &            $0.131$    \\
\hline%----------------------------------------
  $6$                 &            $0.142$    \\
\hline%----------------------------------------
  $7$                 &            $0.149$    \\
\hline%----------------------------------------
  $8$                 &            $0.158$    \\
\hline%----------------------------------------
  $9$                 &            $0.169$    \\
\hline%----------------------------------------
  $10$                &            $0.180$    \\
\hline%----------------------------------------
\end{tabular}
\end{table}

\begin{figure}[H]
  \centering
 	%Trim margins @:   left        bottom       right       top
 	\adjustbox{ trim = {.15\width} {.30\height} {.15\width} {.30\height}, clip }
  {
    \includegraphics[width=\textwidth]{figures/motorConstant.pdf}
  }
	\caption{Plot of test results}
\end{figure}