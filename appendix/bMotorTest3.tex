\pagebreak
\section{Motor Tests - Tachometer Constant} %\label{put a label here and uncomment}
\textbf{Name: Group 510}\\
\textbf{Date: 30/09 - 2015}

\subsubsection{Purpose}
The purpose of the test is to measure and verify that the tachometer constant is 0,030\si{V} multiplied by the motor velocity in radians per second.

\subsubsection{Setup}
\begin{figure}[H]
  \centering
	\includegraphics[scale=0.5]{figures/MotorTest3.pdf}
	\caption{Setup Diagram}
	\flushleft
\end{figure}

\subsubsection{List of Equipment}

\begin{table}[H]
\begin{tabular}{|l|l|p{4cm}|}
\hline%--------------------------------------------------------------------------------
  \textbf{Instrument}                    &  \textbf{AAU-no.}  &  \textbf{Type}       \\
\hline%--------------------------------------------------------------------------------
  Power Supply ($0 - 32$ V) ($0 - 10$ A) &  77076             &  Ea - ps 7032 - 100  \\
\hline%--------------------------------------------------------------------------------
  Multimeter                             &  60764             &  Fluke 189 True RMS  \\
\hline%--------------------------------------------------------------------------------
  Optical tachometer                     &  08246             &  Shimpo DT-205       \\
\hline%--------------------------------------------------------------------------------
\end{tabular}
\end{table}

\subsubsection{Procedure}

\begin{enumerate}
  \item Adjust the voltage of the power supply so that the multimeter measures $6$ V over the tachometer.
  \item Measure the RPM with the Optical tachometer.
\end{enumerate}

\subsubsection{Results}
The tachometer measured 1933 RPM at 6 V, which is used to verify a tachometer constant of \SI{0,03}:
%
\begin{flalign}
 \eq{\frac{1933}{60} \cdot 2 \cdot \pi \cdot \num{0,03}}{\num{6,07}} \approx 6 \ \si{V}&
  \label{eqTachometerConstant}
\end{flalign}