\pagebreak
\section{Motor Tests - Motor Constant} \label{app:motorTestMotorConstant}
\textbf{Name: Group 510}\\
\textbf{Date: 30/09 - 2015}

\subsubsection{Purpose}
The purpose of this of this test is to measure the motor constant \si{K_t}.

\subsubsection{Setup}
\begin{figure}[H]
  \centering
	\includegraphics[scale=0.5]{figures/MotorTest5.pdf}
	\caption{Setup diagram}
\end{figure}

\subsubsection{List of Equipment}

\begin{table}[H]
\begin{tabular}{|l|l|p{4cm}|}
\hline%------------------------------------------------------------------------------------
  \textbf{Instrument}                        &  \textbf{AAU-no.}  &  \textbf{Type}       \\
\hline%------------------------------------------------------------------------------------
  Multimeter 1                               &  60764             &  Fluke 189 True RMS  \\
\hline%------------------------------------------------------------------------------------
  Multimeter 2                   		         &  60769             &  Fluke 189 True RMS  \\
\hline%------------------------------------------------------------------------------------
  Power Supply ($0 - 32$ V) ($0 - 10$ A)     &  77076             &  Ea - ps 7032 - 100  \\
\hline%------------------------------------------------------------------------------------
  Torque sensor                              &  08772             &  Icom                \\
\hline%------------------------------------------------------------------------------------
\end{tabular}
\end{table}

\subsubsection{Procedure}

\begin{enumerate}
  \item Connect the motor shaft to the torque sensor.
  \item Turn on the two multimeter in current and voltage mode respectively.
  \item Start by setting the power supply at 1 A current limiting.
  \item Turn on the supply and note the voltage across the torque sensor.
  \item Repeat the previous two steps up to 10 A with 1 A increments.
\end{enumerate}

\subsubsection{Results}

\begin{table}[H]
\begin{tabular}{|l|l|l|}
\hline%-----------------------------------------
  \textbf{Input (A)}  & \textbf{Output (V)}  \\
\hline%-----------------------------------------
  $1$                 &            0,107    \\
\hline%-----------------------------------------
  $2$                 &            0,113    \\
\hline%-----------------------------------------
  $3$                 &            0,118    \\
\hline%-----------------------------------------
  $4$                 &            0,122    \\
\hline%-----------------------------------------
  $5$                 &            0,131    \\
\hline%-----------------------------------------
  $6$                 &            0,142    \\
\hline%-----------------------------------------
  $7$                 &            0,149    \\
\hline%-----------------------------------------
  $8$                 &            0,158    \\
\hline%-----------------------------------------
  $9$                 &            0,169    \\
\hline%-----------------------------------------
  $10$                &            0,180    \\
\hline%-----------------------------------------
\end{tabular}
\end{table}

\begin{figure}[H]
  \centering
 	%Trim margins @:   left        bottom       right       top
 	\adjustbox{ trim = {.15\width} {.30\height} {.15\width} {.30\height}, clip }
  {
    \includegraphics[width=\textwidth]{figures/motorConstant.pdf}
  }
	\caption{A plot of the torque at different currents, where the blue dots is the measurements and the red line is the least square line.}
	\label{motorConstant}
\end{figure}

The voltages measured are scaled by factor of 0,2 since the torque sensor outputs is 5 V per \si{100N\cdot cm} giving a torque of 0,2 \si{N\cdot m \cdot V^{-1}} \cite{MWAW81P}.
After scaling the voltages to the torques, the measurement is plotted and a least square line is added as seen on \figref{motorConstant}. The relation between the current and torque is described as follows:

\begin{flalign}
   \eq{\tau}{K_t \cdot I_a}\unit{N\cdot m}\nonumber\\
   \eq{K_t}{\frac{\tau}{I_a}}\unit{N\cdot m \cdot A^{-1}}\nonumber
\end{flalign}
\hspace{6mm} Where:\\
\begin{tabular}{p{1cm}lll}
  & \si{\tau}   & is the torque                        &\unitWh{N\cdot m}\\
  & \si{I_a}    & is the current supplied to the motor &\unitWh{A}\\
  & \si{K_t}    & is the motor constant                &\unitWh{N\cdot m \cdot A^{-1}}
\end{tabular}

The value of \si{K_t} is then extracted directly as the slope of the least square regression:
\begin{flalign}
  \eq{K_t}{\num{0,0016}} \ \si{N\cdot m \cdot A^{-1}}&\nonumber
\end{flalign}