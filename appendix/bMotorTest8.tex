\pagebreak
\subsection{Time Constant and Gain} %\label{put a label here and uncomment}
\textbf{Name: Group 510}\\
\textbf{Date: 30/09 - 2015}

\subsubsection{Purpose}
The purpose of this test is to find the motors time constant $\tau$ and gain. This is done by measuring the motors step response.

\subsubsection{Setup}
\begin{figure}[H]
  \centering
	\includegraphics[scale=0.5]{figures/MotorTest8.png}
	\caption{Setup diagram}
\end{figure}

\subsubsection{List of Equipment}

\begin{table}[H]
\begin{tabular}{|l|l|p{4cm}|}
\hline%------------------------------------------------------------------------------------
  \textbf{Instrument}                        &  \textbf{AAU-no.}  &  \textbf{Type}       \\
\hline%------------------------------------------------------------------------------------
  Oscilloscope                               &  64672             &  Agilent DSO6034A    \\
\hline%------------------------------------------------------------------------------------
  Power Supply ($0 - 32$ V) ($0 - 10$ A)     &  77076             &  Ea - ps 7032 - 100  \\
\hline%------------------------------------------------------------------------------------
  Optical tachometer                         &  77087             &  Compact             \\
\hline%------------------------------------------------------------------------------------
\end{tabular}
\end{table}

\subsubsection{Procedure}

\begin{enumerate}
  \item Turn on the oscilloscope, and connect one channel to the power supply, and another to the tachometer.
  \item On the oscilloscope press the "trigger mode"-key choose the "normal"-option, set the trigger to "rising-edge", and the trigger source to the channel connected to the power supply.
  \item To prevent false triggering on the oscilloscope set the trigger value to \num{4,50}V with the turn-key.
  \item Press "single"-key on oscilloscope.
  \item Turn on the power supply at 5 volt.
  \item Insert a USB-flash drive in the oscilloscope and press the save key to extract the data.
\end{enumerate}

\subsubsection{Results}

\begin{figure}[H]
  \centering
  \includegraphics[width=.8\textwidth]{figures/mechanicalTimeConstant.pdf}
	\caption{A plot of the step response illustrating the angular velocity over time. The blue dots is the measurements and the red line indicates the mechanical time constant.}
	\label{mechanicalTimeConstant}
\end{figure}

The graph in \figref{mechanicalTimeConstant} shows the angular velocity of the motor over time. The read line shows the time constant at \si{\num{63.2} \%} of max angular velocity. In the data the mechanical time constant is found to be:
%
\begin{flalign}
  \eq{\tau_{mec}}{0,238} \tx{ s}&&\nonumber
\end{flalign}

\begin{figure}[H]
  \centering
 	\includegraphics[width=.8\textwidth]{figures/motorGain.pdf}
  \caption{A plot of the motor's step response, where the input is motor voltage and the output is the angular velocity. The blue dots indicates the measurements and the red line is the tendency line.}
	\label{motorGain}
\end{figure}
%
The graph in \figref{motorGain} shows motor voltage in relation to the angular velocity, which reveals the gain, \si{K} of the system, as the slope of the least square regression line:
%
\begin{flalign}
  \eq{K}{569.19} \si{\ rad \cdot s^{-1} \cdot V^{-1}}&\nonumber
\end{flalign}