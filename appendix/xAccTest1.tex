\pagebreak
\section{Accept Test - Communication link} \label{app:AccTes1}
\textbf{Name: Group 510}\\
\textbf{Date: 11/12 - 2015}

\subsubsection{Purpose}
The purpose of the test, is to prove, that the vehicle gets the data from the GOT system.

\subsubsection{List of Equipment}
\begin{table}[H]
\begin{tabular}{|l|l|p{4cm}|}
\hline%------------------------------------------------------------------------------------
  \textbf{Instrument}                        &  \textbf{AAU-no.}  &  \textbf{Type}       \\
\hline%------------------------------------------------------------------------------------
  GOT system                               &   &  Indoor position system  \\
\hline%------------------------------------------------------------------------------------
  Computer                   		         &   &    \\
\hline%------------------------------------------------------------------------------------
  2 Xbee &  -             &    \\
\hline%------------------------------------------------------------------------------------
  Explore board to Xbee                 &  -             &                      \\
\hline
 Arduino & & Mega 2560 \\
\hline
%------------------------------------------------------------------------------------
\end{tabular}
\end{table}

\subsubsection{Procedure}

\begin{enumerate}
  \item Setup GOT system, by power up the satellites and the master.
  \item Place the GOT transmitter onto the vehicle and turn it on.
  \item Start the GOT code, which contains the protocol.
  \item Turn on the vehicle.
  \item Check with a serial port on the arduino, to look for data.
  \item Save the data from the serial port on the arduino.
  \item Find the log for the data send from the GOT system, in the GOT folder.
\end{enumerate}

\subsubsection{Results}

\begin{table}[H]
\begin{tabular}{|c||c||c|}
\hline
Transmitted & Received & Control \\
\hline
240,128,224,26,192,5,240,33,216,24,202,15 & 240,128,224,26,192,5,240,33,216,24,202,15 & True \\
\hline
240,128,224,26,0,6,16,33,216,230,201,15 & 240,128,224,26,0,6,16,33,216,230,201,15 & True \\
\hline
240,128,224,26,0,6,176,33,216,220,201,15 & 240,128,224,26,0,6,176,33,216,220,201,15 & True \\
\hline
240,128,224,27,0,6,112,33,216,224,185,15 & 240,128,224,27,0,6,112,33,216,224,185,15 & True \\
\hline
240,128,224,27,192,5,176,33,216,28,186,15 & 240,128,224,27,192,5,176,33,216,28,186,15 & True \\
\hline
240,128,96,28,192,5,80,34,216,18,178,15 & 240,128,96,28,192,5,80,34,216,18,178,15 & True \\
\hline
240,128,224,28,128,5,80,34,216,82,170,15 & 240,128,224,28,128,5,80,34,216,82,170,15 & True \\
\hline
240,128,224,29,64,5,80,34,216,146,154,15 & 240,128,224,29,64,5,80,34,216,146,154,15 & True \\
\hline
240,128,224,28,128,5,208,33,216,90,170,15 & 240,128,224,28,128,5,208,33,216,90,170,15 & True \\
\hline
240,128,224,27,192,5,176,33,216,28,186,15 & 240,128,224,27,192,5,176,33,216,28,186,15 & True \\
\hline
240,128,96,29,64,5,208,33,216,154,162,15 & 240,128,96,29,64,5,208,33,216,154,162,15 & True \\
\hline
240,128,224,28,128,5,16,34,216,86,170,15 & 240,128,224,28,128,5,16,34,216,86,170,15 & True \\
\hline
240,128,96,30,0,5,240,33,216,216,146,15 & 240,128,96,30,0,5,240,33,216,216,146,15 & True \\
\hline
240,128,224,28,128,5,16,34,216,86,170,15 & 240,128,224,28,128,5,16,34,216,86,170,15 & True \\
\hline
240,128,224,28,64,5,48,34,216,148,170,15 & 240,128,224,28,64,5,48,34,216,148,170,15 & True \\
\hline
240,128,224,29,64,5,48,34,216,148,154,15 & 240,128,224,29,64,5,48,34,216,148,154,15 & True \\
\hline
240,128,224,27,192,5,208,33,216,26,186,15 & 240,128,224,27,192,5,208,33,216,26,186,15 & True \\
\hline
240,128,224,28,128,5,176,33,216,92,170,15 & 240,128,224,28,128,5,176,33,216,92,170,15 & True \\
\hline
240,128,96,28,128,5,144,34,216,78,178,15 & 240,128,96,28,128,5,144,34,216,78,178,15 & True \\
\hline
240,128,224,28,128,5,176,34,216,76,170,15 & 240,128,224,28,128,5,176,34,216,76,170,15 & True \\
\hline
240,128,224,28,128,5,112,34,216,80,170,15 & 240,128,224,28,128,5,112,34,216,80,170,15 & True \\
\hline
240,128,224,28,64,5,208,34,216,138,170,15 & 240,128,224,28,64,5,208,34,216,138,170,15 & True \\
\hline
240,128,224,29,0,5,112,34,216,208,154,15 & 240,128,224,29,0,5,112,34,216,208,154,15 & True \\
\hline
240,128,224,28,64,5,144,34,216,142,170,15 & 240,128,224,28,64,5,144,34,216,142,170,15 & True \\
\hline
240,128,224,28,128,5,16,34,216,86,170,15 & 240,128,224,28,128,5,16,34,216,86,170,15 & True \\
\hline
\end{tabular}
\caption{The packages send from GOT to the arduino}
\label{AccTest1Tab}
\end{table}

As shown in \tableref{AccTest1Tab}, the transmission to the receiver is working and all 25 packages are send and received the same. With the help of the protocol, explain in \secref{}, these packages can be converted back to the three coordinates, shown in \tableref{AccTest1Tab2}.

\begin{table}[H]
\begin{tabular}{|c||c||c|}
\hline
Transmitted & Received & Control \\
\hline
\end{tabular}
\caption{The data send from GOT to the arduino}
\label{AccTest1Tab2}
\end{table}

\subsubsection{Conclusion}
