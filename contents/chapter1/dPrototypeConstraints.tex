\section{Prototype Constraints}
The aim of this project is to create a functional prototype of an automated lawn mower. To provide proof of concept not all the described functionalities are included. The following sections include a brief description of the technology base on which the prototype is constructed, along with argumentation for delimitation of functionalities.

\subsection{Technology Base}
The prototype base is a tracked vehicle as seen on \figref{TrackedVehicle}, it comes with a brushed DC motor which provides drive power for the belts and a servo which breaks on each side to control the ratio of the differential steering. The tracked vehicle also includes two hall sensors, one by each belt, which keeps track of the speed by measuring pulses from magnets mounted on the front wheels.

\begin{figure}[H]
	\centering
	\includegraphics[scale=.7]{figures/aaulogo-en}
	\flushleft
	\caption{Tracked vehicle with brushed DC motor and a servo\fxnote{Insert picture of technology base}}
	\label{TrackedVehicle}
\end{figure}

\subsection{Grass Length Detection}
Detection of the grass length to control the speed of the lawn mower thus ensuring an evenly cut lawn is a submodule which can be added at any time. Since it is not fatal for a working system and might even be unnecessary depending on time between each mowing of the lawn, it is decided to exclude this functionality from the initial design.

\subsection{Rain Sensor}
As the lawn mower is supposed to work outside, it is important to consider that it may be raining, and that an electronic device can be affected by those environmental issues. Even if the electrical part is waterproofed, there is a mechanical threat, a rain sensor could be a security warning to order the vehicle to get back in time.
It is possible to build the vehicle aware of those issues and add mechanical modules to secure it. However, the prototype will only be tested indoor, so this type of sensor will not be necessary.

\subsection{Obstacle Avoidance}
The lawn movers path might not always be clear, there could be some trees, garden tools, tables and chair, or even persons walking in it. The vehicle should be aware of what is in front of it at any time, to correct its path and get around the obstacle. To avoid this the sensor could be a pushing button to detect a solid strong object, or an ultrasound detector if the object is breakable.
As the aim of the project is to control the path of the vehicle by using angular positioning sensors, a proximity sensor will not be included. Statics objects could be registered on the map to avoid these cases.

\subsection{Power Monitorng}
Power monitoring could be implemented by measuring the voltage across the batteries, to be sure that the lawn mower is not running out of power, and have enough to go back to the station. 
This will not concern this project because it will be used as an indoor prototype, and in worse case, the last position of the vehicle will be recorded. The user can then take it by hand and plug it into the station.